\documentclass[11pt, justified]{tufte-book}
\usepackage{amsmath, amssymb, amsthm, bm, moresize}
\usepackage{euler}

\usepackage{graphicx, epsdice, xcolor, listings, float, wrapfig, caption}
\geometry{
  left=1in, % left margin
  textwidth=5.0in, % main text block
  marginparsep=0.2in, % gutter between main text block and margin notes
  marginparwidth=1.8in % width of margin notes
}

\usepackage{etoolbox}
\patchcmd{\thebibliography}{\section*{\refname}}{}{}{}

\newcommand{\EE}{\mathbb{E}}
\newcommand{\PP}{\mathbb{P}}
\newcommand{\RR}{\mathbb{R}}
%
\newcommand{\Dd}{\mathcal{D}}
\newcommand{\Ee}{\mathcal{E}}
\newcommand{\Ff}{\mathcal{F}}
\newcommand{\Gg}{\mathcal{G}}
\newcommand{\Hh}{\mathcal{H}}
\newcommand{\Ii}{\mathcal{I}}
\newcommand{\Kk}{\mathcal{K}}
\newcommand{\Ll}{\mathcal{L}}
\newcommand{\Mm}{\mathcal{M}}
\newcommand{\Nn}{\mathcal{N}}
\newcommand{\Ss}{\mathcal{S}}
\newcommand{\Tt}{\mathcal{T}}
\newcommand{\Uu}{\mathcal{U}}
\newcommand{\Vv}{\mathcal{V}}
\newcommand{\Xx}{\mathcal{X}}
\newcommand{\Yy}{\mathcal{Y}}

\newtheorem*{qst}{Question}
\newtheorem*{thm}{Theorem}
\newtheorem*{lem}{Lemma}
\newtheorem*{prop}{Proposition}
\newtheorem*{clm}{Claim}
\theoremstyle{definition}
\newtheorem*{dfn}{Definition}

\definecolor{mblu}{rgb}{0.05, 0.40, 0.70}
\newcommand{\mtit}[1]{%
    {%
        \noindent
        \Huge\color{mblu}\bf\textsf{#1}
        \vspace{0.3cm}
    }
}
\newcommand{\msec}[1]{\section{\color{mblu}\textsf{#1}}}

\begin{document}

    \mtit{The Classification of Surfaces}
    \msec{Bird's eye view}
        The classification of manifolds presents a grand challenge.  To begin
        addressing it, we must decide on a notion of \textit{sameness}.  For
        smooth manifolds, such notions range in strength from homotopy
        equivalence to diffeomorphism.  Do these notions coincide?
        %
        For instance, we like to distinguish compact (connected, smooth)
        surfaces by their Euler characteristics.  Does this tool miss anything?
        That is, are there non-diffeomorphic compact surfaces with the same
        $\chi$? 
        %
        The answer is no: \textit{the Euler characteristic determines compact
        surfaces up to diffeomorphism}.

        We may establish this result in several ways.  For example, we use
        smoothness to construct a triangulation, whereupon we reason
        combinatorially to bring a surface to standard form.  Or, we could use
        smoothness to construct a metric, whereupon we reason
        geometrically to bring the surface to a standard form.
        Just as I regard finite differences to be messier than derivatives, I
        find the geometric approach more symmetrical than the combinatorial
        one.  In what follows, we'll describe this approach.

        Any smooth manifold $\Mm$ embeds smoothly into some standard sphere and
        therefore enjoys a smooth Riemannian metric $g$.\footnote{
            Intuitively, the metric corresponds to an infinitely fine
            triangulation; it induces a natural volume form measuring the
            density of highest-dimensional cells (and likewise on each subspace
            of each tangent space).  Refining triangulations thus becomes as
            easy as scaling the metric.
        }
        We aim to rescale the metric so that it has constant curvature.  The
        point is that we may then compare our surface $\Mm$ to a standard
        surface.  Indeed, constant curvature implies that $\Mm$ is locally
        isometric to a standard (simply connected) model space $\Ss$.  By
        patching together these local isometries, we isometrically identify
        $\Ss$ with the universal cover of $\Mm$.  Since both that isometry and
        the the deck group act by diffeomorphisms, $\Mm$ itself has the smooth
        structure of a standard quotient of $\Ss$.  We finish by recalling that
        such standard surfaces are determined by their Euler characteristic. 

    \msec{Ricci flow}

        So let's focus on the problem of finding a metric of constant
        curvature.  We'll do this by evolving our initial metric by a sort of
        diffusion equation.
        Let's see how to do this in the special case where
        the initial metric $g$ has (potentially varying) negative
        curvature everywhere.  We'd like the regions with extreme negative
        curvature to expand, thus diluting their curvature.  Conversely, we'd
        like regions that are too flat to contract, thus concentrating their
        curvature.  So, writing $R$ for the scalar curvature (scaled by
        $1/8\pi^2$ so that it averages to $\chi$) let us scale the metric
        according to how much $R$ overestimates its average:\footnote{
            Intuitively, we split octogonal singularities into pairs of
            heptagonal singularities so that the graph becomes increasingly
            regular.
        }
        $$
            \dot{g} = (R - \chi) g
        $$
        Note that this equation is non-linear.  Intuitively, the non-linearity
        indicates the creation of new (positive and negative) curvature due to
        the non-uniformity of $g$'s scaling.  This ``pair production''
        complicates our goal of diffusion, but we will see that in $2$
        dimensions, diffusion prevails.

        It's enough to find a solution $g(t,x)$, defined for all time, that
        converges to a fixed point.  To show long-term existence, it's enough
        to bound $R(t,x)$ over all time and space.



    \msec{The curvature is never too small}
        
    \msec{The curvature is never too big}
        

\end{document}


