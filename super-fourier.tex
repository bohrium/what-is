% author:   sam tenka
% create:   2022-02
% change:   2022-02

%==============================================================================
%=====  0.  DOCUMENT SETTINGS  ================================================
%==============================================================================

%~~~~~~~~~~~~~~~~~~~~~~~~~~~~~~~~~~~~~~~~~~~~~~~~~~~~~~~~~~~~~~~~~~~~~~~~~~~~~~
%~~~~~~~~~~~~~  0.0. About and Beyond this Exposition  ~~~~~~~~~~~~~~~~~~~~~~~~

%---------------------  0.0.0. page geometry  ---------------------------------
\documentclass[12pt]{article}
\usepackage[top=3cm, bottom=3cm, left=3cm, right=3cm]{geometry}

%---------------------  0.0.1. math packages  ---------------------------------
\usepackage{amsmath, amssymb, amsthm, mathtools, bm, moresize, euler}

%---------------------  0.0.2. graphics packages  -----------------------------
\usepackage{graphicx, epsdice, xcolor, float, wrapfig, caption}

%~~~~~~~~~~~~~~~~~~~~~~~~~~~~~~~~~~~~~~~~~~~~~~~~~~~~~~~~~~~~~~~~~~~~~~~~~~~~~~
%~~~~~~~~~~~~~  0.1. Header Formatting  ~~~~~~~~~~~~~~~~~~~~~~~~~~~~~~~~~~~~~~~

%---------------------  0.1.0. section header style  --------------------------
\definecolor{mblu}{rgb}{0.05, 0.40, 0.70}
\newcommand{\msec}[1]{\subsection*{\color{mblu}\textsf{#1}}}

%---------------------  0.1.1. clear the bibliography's header  ---------------
\usepackage{etoolbox}
\patchcmd{\thebibliography}{\section*{\refname}}{}{}{}

%~~~~~~~~~~~~~~~~~~~~~~~~~~~~~~~~~~~~~~~~~~~~~~~~~~~~~~~~~~~~~~~~~~~~~~~~~~~~~~
%~~~~~~~~~~~~~  0.2. Math Symbols and Blocks  ~~~~~~~~~~~~~~~~~~~~~~~~~~~~~~~~~

\newcommand{\KL}{\text{KL}}
\newcommand{\EN}{\text{H}}

%---------------------  0.2.0. caligraphic letters  ---------------------------
\newcommand{\Dd}{\mathcal{D}}
% ...
\newcommand{\Hh}{\mathcal{H}}
% ...
\newcommand{\Ll}{\mathcal{L}}
\newcommand{\Mm}{\mathcal{M}}
\newcommand{\Nn}{\mathcal{N}}
\newcommand{\Oo}{\mathcal{O}}
\newcommand{\Pp}{\mathcal{P}}
% ...
\newcommand{\Ss}{\mathcal{S}}
% ...
\newcommand{\Xx}{\mathcal{X}}

%---------------------  0.2.1. double struck letters  -------------------------
\newcommand{\CC}{\mathbb{C}}
\newcommand{\DD}{\mathbb{D}}
\newcommand{\EE}{\mathbb{E}}
% ...
\newcommand{\NN}{\mathbb{N}}
\newcommand{\OO}{\mathbb{O}}
\newcommand{\PP}{\mathbb{P}}
\newcommand{\QQ}{\mathbb{Q}}
\newcommand{\RR}{\mathbb{R}}
% ...
\newcommand{\ZZ}{\mathbb{Z}}

%---------------------  0.2.2. math environments  -----------------------------
\newtheorem*{qst}{Question}
\newtheorem*{thm}{Theorem}
\newtheorem*{lem}{Lemma}
% ...
\theoremstyle{definition}
\newtheorem*{dfn}{Definition}

%~~~~~~~~~~~~~~~~~~~~~~~~~~~~~~~~~~~~~~~~~~~~~~~~~~~~~~~~~~~~~~~~~~~~~~~~~~~~~~
%~~~~~~~~~~~~~  0.3. Title  ~~~~~~~~~~~~~~~~~~~~~~~~~~~~~~~~~~~~~~~~~~~~~~~~~~~

\begin{document}
{
    \centering \Huge \sf \color{mblu} 
    Types considered harmful
    \vspace{0.5cm}
}

    There are some who argue that powerful type systems help 
        to express complex interfaces parsimoniously,
        to aid verification by human and computer, and
        to automate deep optimizations. 
    %
    But how far is too far?  How much of the sentimental touch of from-scratch
    \texttt{C} is one willing to forgo in pursuit of ``correctness''? 
    %

    We are now eager to present \texttt{Typo}, our 



{
    \centering \Huge \sf \color{mblu} 
    A Short Note on Timbre
    \vspace{0.5cm}
}

    Interestingly, the fourier uncertainty bound does not limit human hearing:
    the bound constrains the a spectral line's width, which is not the same as
    our error in estimating that line's center.  In fact, a meticulous poll of
    roommates ($N=2$) establishes that humans may hear gaussian-enveloped
    pitches at a resolution of at least $(\pm 10\text{ms}) \times \pm 2 Hz =
    0.02$, an order of magnitude finer than the natural symplectic scale
    $1/2\pi \approx 0.16$. 

    We present two techniques suited to spectral analysis of music.

%==============================================================================
%=====  1.  INTRO  ============================================================
%==============================================================================

%==============================================================================
%=====  6.  BACKMATTER  =======================================================
%==============================================================================

\end{document}


