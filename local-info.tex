% author:   sam tenka
% create:   2022-02
% change:   2022-02

%==============================================================================
%=====  0.  DOCUMENT SETTINGS  ================================================
%==============================================================================

%~~~~~~~~~~~~~~~~~~~~~~~~~~~~~~~~~~~~~~~~~~~~~~~~~~~~~~~~~~~~~~~~~~~~~~~~~~~~~~
%~~~~~~~~~~~~~  0.0. About and Beyond this Exposition  ~~~~~~~~~~~~~~~~~~~~~~~~

%---------------------  0.0.0. page geometry  ---------------------------------
\documentclass[12pt]{article}
\usepackage[top=3cm, bottom=3cm, left=3cm, right=3cm]{geometry}

%---------------------  0.0.1. math packages  ---------------------------------
\usepackage{amsmath, amssymb, amsthm, mathtools, bm, moresize, euler}

%---------------------  0.0.2. graphics packages  -----------------------------
\usepackage{graphicx, epsdice, xcolor, float, wrapfig, caption}

%~~~~~~~~~~~~~~~~~~~~~~~~~~~~~~~~~~~~~~~~~~~~~~~~~~~~~~~~~~~~~~~~~~~~~~~~~~~~~~
%~~~~~~~~~~~~~  0.1. Header Formatting  ~~~~~~~~~~~~~~~~~~~~~~~~~~~~~~~~~~~~~~~

%---------------------  0.1.0. section header style  --------------------------
\definecolor{mblu}{rgb}{0.05, 0.40, 0.70}
\newcommand{\msec}[1]{\subsection*{\color{mblu}\textsf{#1}}}

%---------------------  0.1.1. clear the bibliography's header  ---------------
\usepackage{etoolbox}
\patchcmd{\thebibliography}{\section*{\refname}}{}{}{}

%~~~~~~~~~~~~~~~~~~~~~~~~~~~~~~~~~~~~~~~~~~~~~~~~~~~~~~~~~~~~~~~~~~~~~~~~~~~~~~
%~~~~~~~~~~~~~  0.2. Math Symbols and Blocks  ~~~~~~~~~~~~~~~~~~~~~~~~~~~~~~~~~

%---------------------  0.2.0. caligraphic letters  ---------------------------
\newcommand{\Dd}{\mathcal{D}}
% ...
\newcommand{\Hh}{\mathcal{H}}
% ...
\newcommand{\Ll}{\mathcal{L}}
\newcommand{\Mm}{\mathcal{M}}
\newcommand{\Nn}{\mathcal{N}}
\newcommand{\Oo}{\mathcal{O}}
\newcommand{\PP}{\mathcal{P}}
% ...
\newcommand{\Ss}{\mathcal{S}}
% ...
\newcommand{\Xx}{\mathcal{X}}

%---------------------  0.2.1. double struck letters  -------------------------
\newcommand{\CC}{\mathbb{Z}}
% ...
\newcommand{\NN}{\mathbb{N}}
% ...
\newcommand{\QQ}{\mathbb{Q}}
\newcommand{\RR}{\mathbb{R}}
% ...
\newcommand{\ZZ}{\mathbb{Z}}

%---------------------  0.2.2. math environments  -----------------------------
\newtheorem*{qst}{Question}
\newtheorem*{thm}{Theorem}
\newtheorem*{lem}{Lemma}
% ...
\theoremstyle{definition}
\newtheorem*{dfn}{Definition}

%~~~~~~~~~~~~~~~~~~~~~~~~~~~~~~~~~~~~~~~~~~~~~~~~~~~~~~~~~~~~~~~~~~~~~~~~~~~~~~
%~~~~~~~~~~~~~  0.3. Title  ~~~~~~~~~~~~~~~~~~~~~~~~~~~~~~~~~~~~~~~~~~~~~~~~~~~

\begin{document}

{
    \centering \Huge \sf \color{mblu} 
    Detailed Local KL Geometry
    \vspace{0.5cm}
}

%==============================================================================
%=====  1.  INTRO  ============================================================
%==============================================================================

    The local KL geometry of statistical manifolds deviates from euclidean
    geometry in two ways: its ``metric'' has \emph{superquadratic} terms,
    most notably cubic (asymmetry) and quartic (barrier); and its quadratic
    component induces a \emph{curved} connection.
    %
    More precisely, we have a divergence $D:\Mm\times \Mm \to [0,\infty]$ 
    that vanishes on and only on the diagonal; $D$ is smooth on a neighborhood
    of this diagonal; on this diagonal, the hessian is invariant with respect
    to interchange of the two $\Mm$ factors and otherwise generic.
    With $v=(a,b)$, we thus write:
    $$
        D(p+a:p+b) = (1/2)H_p(v,v) + (1/6)J_p(v,v,v) + (1/24)Q_p(v,v,v,v) 
                     + o(v^{\otimes 4})
    $$
    Here, $H$ is positive definite and $H,J,Q$ --- as well as the little $o$
    constant --- vary smoothly with $p$. 
    %
    The situation has a hidden symmetry under interchange of $v$ with $\bar v =
    (b,a)$.  For example, $H(v,v) = H(\bar v, \bar v)$. 
    %
    To see how this arises, let us focus on a submanifold $\Mm \subseteq
    \Delta^o(X)$ of the open simplex on a finite set $X$.  Since $\log(1+z) =
    z-z^2/2+z^3/3-z^4/4+\cdots$ and
    $1/(p+b) = (1/p)(1-(b/p)+(b/p)^2-(b/p)^3+\cdots)$,
    we have
    \begin{align*}
          &D(p+a, p+b) \\
        %= &\sum (p+a) \log\left(\frac{p + a}{p + b}\right)\\ 
        = &\sum (p+a) \log\left(1 + \frac{a-b}{p + b}\right)\\ 
        = &\sum (p+a) \left(
                        \left(\frac{a-b}{p + b}\right)
            -\frac{1}{2}\left(\frac{a-b}{p + b}\right)^2
            +\frac{1}{3}\left(\frac{a-b}{p + b}\right)^3
            -\frac{1}{4}\left(\frac{a-b}{p + b}\right)^4
        \right)\\
        = &\sum p(1+\alpha) \left(
                        \delta   (1 -\beta + \beta^2 - \beta^3)
            -\frac{1}{2}\delta^2 (1 -2\beta + 3\beta^2)
            +\frac{1}{3}\delta^3 (1 -3\beta)
            -\frac{1}{4}\delta^4 
        \right)
    \end{align*}
    Here, $\alpha=a/p$, $\beta=b/p$, $\gamma=\alpha+\beta$, $\delta = \alpha-\beta$.
    We expand to:
    \begin{align*}
        =  \sum p \delta (1+\alpha) (
              \left(1\right)  
            +&\left(-\beta-\delta/2\right)\\
            +&\left(\beta^2 +2\delta\beta/2 +\delta^2/3\right)\\
            +&\left(-\beta^3-3\delta\beta^2/2-3\delta^2\beta/3-\delta^3/4\right) )
    \end{align*}
    The degree one sum vanishes by normalization, as expected. 
    The degree two term has the familiar $\chi^2$ form:
    $$
        p\delta (-\beta-\delta/2 + \alpha) = p \delta^2/2
    $$
    The degree three term witnesses asymmetry:
    $$
        p\delta (\beta^2 +2\delta\beta/2 +\delta^2/3 -\alpha\beta-\alpha\delta/2)
        = p(-(5/12)\delta^3  + \gamma\delta^2/4)
    $$
    The degree four term is:
    $$
        p\delta (-\beta^3-3\delta\beta^2/2-3\delta^2\beta/3-\delta^3/4   
                 +\alpha\beta^2 +2\alpha\delta\beta/2 +\alpha\delta^2/3)
        %= p\delta(\delta^3/4 - 2abb + (5/6)aab -(5/6)aaa) 
        = p(\delta^4/4 - (5/6)\alpha^2\delta^2 - 2\alpha\beta^2\delta) 
    $$


    We now explore the basics of this geometry. 
    %
    We especially examine how the new geometry distorts our euclidean picture
    of a large-$N$ sample as a tight Gaussian on an inner product space.

    \msec{}
    \msec{}

%==============================================================================
%=====  6.  BACKMATTER  =======================================================
%==============================================================================

\end{document}


