\documentclass[11pt, justified]{tufte-book}
\usepackage{amsmath, amssymb, amsthm, bm, moresize}
\usepackage{euler, tikz-cd}

\usepackage{graphicx, epsdice, xcolor, listings, float, wrapfig, caption}
\geometry{
  left=1in, % left margin
  textwidth=5.0in, % main text block
  marginparsep=0.2in, % gutter between main text block and margin notes
  marginparwidth=1.8in % width of margin notes
}

\usepackage{etoolbox}
\patchcmd{\thebibliography}{\section*{\refname}}{}{}{}

\newcommand{\EE}{\mathbb{E}}
\newcommand{\PP}{\mathbb{P}}
\newcommand{\RR}{\mathbb{R}}
\newcommand{\ZZ}{\mathbb{Z}}
%
\newcommand{\Dd}{\mathcal{D}}
\newcommand{\Ee}{\mathcal{E}}
\newcommand{\Ff}{\mathcal{F}}
\newcommand{\Gg}{\mathcal{G}}
\newcommand{\Hh}{\mathcal{H}}
\newcommand{\Ii}{\mathcal{I}}
\newcommand{\Kk}{\mathcal{K}}
\newcommand{\Ll}{\mathcal{L}}
\newcommand{\Mm}{\mathcal{M}}
\newcommand{\Nn}{\mathcal{N}}
\newcommand{\Ss}{\mathcal{S}}
\newcommand{\Tt}{\mathcal{T}}
\newcommand{\Uu}{\mathcal{U}}
\newcommand{\Vv}{\mathcal{V}}
\newcommand{\Xx}{\mathcal{X}}
\newcommand{\Yy}{\mathcal{Y}}

\newtheorem*{qst}{Question}
\newtheorem*{thm}{Theorem}
\newtheorem*{lem}{Lemma}
\newtheorem*{prop}{Proposition}
\newtheorem*{clm}{Claim}
\theoremstyle{definition}
\newtheorem*{dfn}{Definition}

\definecolor{mblu}{rgb}{0.05, 0.40, 0.70}
\newcommand{\mtit}[1]{%
    {%
        \noindent
        \Huge\color{mblu}\bf\textsf{#1}
        \vspace{0.3cm}
    }
}
\newcommand{\msec}[1]{
    \chapter{\color{mblu}\textsf{#1}}
}
\newcommand{\mssec}[1]{\section{\color{mblu}\textsf{#1}}}

\begin{document}
\renewcommand{\cleardoublepage}{}
%\renewcommand{\clearpage}{}

    \mtit{Numerology}

        I've long wanted to take a math class where on the $n$th day we learn
        about the number $n$.\footnote{
            Looking back, I think I was influenced by John Baez' delightful,
            fact-dense talks on the numbers five, eight, and twenty-four. 
        } These notes attempt to give the reader such an
        experience.  Alas, they are by no means complete.

        In particular, we present one perspective per $n$ while ignoring $n$'s
        other important aspects.  Any text on numbers must omit something:
        Guy observes that, since we burden the several small numbers\footnote{
            $n$ so small, say, that we would recognize $n$ apples by sight alone 
        } with too many jobs,
        most small numbers have multiple personalities.  In fact, one of the
        main lessons I learned while an undergraduate was to spend less time
        compressing my understanding toward One True viewpoint and more time
        absorbing Many Clarifying viewpoints.  
            %
            The Fourier transform, for instance, is significant both because it
            diagonalizes translation and because it diagonalizes
            differentiation.  The former viewpoint leads to the representation
            theory of groups; the latter, the harmonic theory of metric
            manifolds.
            %
        Some sheaves simply lack global sections.

        With this in mind, the viewpoints I choose to discuss are these:
        \textbf{Zero} names what we'd like to ignore and hence is an accomplice
        to abstraction.  \textbf{One} demarcates existence and uniqueness and
        so highlights the role of coordinates.  Because \textbf{Two} counts the
        set $\{\text{algebra}, \text{geometry}\}$, quadratic forms pervade
        mathematics.  \textbf{Three}\footnote{also known as Arnold's constant;
        we will ignore this thread of thought}
        is the beginning of complexity, for degree-three graphs are
        non-trivial. 

            Most of these notes I can trace to conversations with friends and
            friendly experts,  especially Dean Young and Andrew Snowden.  I
            hope that these notes provide entertainment despite their
            un-originality.


    \msec{Zero: Analysis through Equality}
        \mssec{Approximation and Abstraction}

        What do the quotient operations $V/W$ of vector spaces and $S/(\sim)$
        of sets have in common?  They are both coequalizers; the former, of
        the two arrows $W\hookrightarrow V$ and $W\to 0\to V$, and the latter,
        of $(\sim)\hookrightarrow S\times S\to S$ by left or right projection.
        For sets, we must specify both sides of the equations $a=b$ to make
        true; for vector spaces, by contrast, we can get away with specifying
        only one side of $a-b=0$ due to the presence of an initial-and-terminal
        object, $0$.  Categories with such a \textbf{zero object} include
        those of groups, based spaces, and Noetherian lattices.

        Naming our ignorance gives us valuable flexibility as we analyze
        structures into simpler parts.  Indeed, the two grand methods of
        simplification are \textbf{approximation}, where we posit
        false-but-useful properties, and \textbf{abstraction}, where we discard
        true-but-useless distinctions.  The concept of $0$ gives us the language
        for both activities, as illustrated in the concept of 
        \textbf{exact sequences} $F\to E\to B$.  These are just solid commuting
        squares
        as below with respect to which $F$ is terminal and $B$ is
        initial.\footnote{
            To be explicit, for every choice of $X,Y$, and dashed arrows that
            commute with the solid arrows, there should exist unique dotted
            arrows to make all commute.
        }
        $$
            \begin{tikzcd}
                                                                           &  &                         & 0 \arrow[rd] \arrow[rrrd, dashed] &                      &  &   \\
            X \arrow[rrru, dashed] \arrow[rrrd, dashed] \arrow[rr, dotted] &  & F \arrow[ru] \arrow[rd] &                                   & B \arrow[rr, dotted] &  & Y \\
                                                                           &  &                         & E \arrow[ru] \arrow[rrru, dashed] &                      &  &  
            \end{tikzcd}
        $$
        Intuitively, each such square analyzes $E$ into conceptually orthogonal
        parts $F, B$.  The universal properties ask $F, B$ to hug $E$ as
        tightly as possible despite the orthogonality imposed by $0$.  This was
        Descartes' great idea: to study Euclid axis-by-axis.

        Studying $F\to E\to B$ in a concrete category, we sense that $F$
        approximates $E$ and $B$ abstracts $E$.\footnote{
            For example, consider the groups
            $$
                \ZZ \to \RR \to S^1
            $$
        } In particular, $F\to E$ represents
        $F$ as a subset of $E$, and to factor $X\to E$ through $X\to F$ is to
        ``round'' decimal outputs in $E$ to integer outputs in $F$.  Dually, $E\to
        B$ presents $B$ as a quotient of $E$, and to factor $E\to Y$ through
        $E\to B$ is to ``collapse'' the decimal inputs in $E$ to their
        fractional parts in $B$.

        As another example, consider the category of parameterized
        distributions $p(Y;X)$ on two real numbers, with maps induced by linear
        shears $y \mapsto ax + by + c$.  Any $p(Y;X)$ induced by a centered
        normal joint distribution enjoys an exact decomposition
        $$
            p_I \to p\to p_E
        $$
        where $p_I(Y;X=x) = p(Y;X=0)$ is an independent approximation to $p$
        and $p_E(Y=y;X) = \delta(y-\EE_p[Y;X])$ is an abstraction of $p$ in
        terms of a summary statistic.

        \mssec{Distributivity}

            Multiplication distributes over addition, but not vice versa.  As a
            consequence we typically have $0\cdot n = 0$ but $1 + n \neq 1$.
            Thus an ideal (i.e.\ the pre-image of $0$ for some ring map) is
            closed under both addition and scalar multiplication, but a
            multiplicative system (i.e.\ the pre-image of $1$ for some ring
            map) is typically closed only under multiplication and not scalar
            addition.

        \mssec{Homology}

    \msec{One: Invariance}
    \msec{Two: the Ghost of Pythagoras}
    \msec{Three: Encapsulation through Action}
        \mssec{Homology}
        blue

\end{document}


