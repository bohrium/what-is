% author:   sam tenka
% create:   2022-02
% change:   2022-02

%==============================================================================
%=====  0.  DOCUMENT SETTINGS  ================================================
%==============================================================================

%~~~~~~~~~~~~~~~~~~~~~~~~~~~~~~~~~~~~~~~~~~~~~~~~~~~~~~~~~~~~~~~~~~~~~~~~~~~~~~
%~~~~~~~~~~~~~  0.0. About and Beyond this Exposition  ~~~~~~~~~~~~~~~~~~~~~~~~

% page geometry: 
\documentclass[twocolumn, 11pt]{article}
\usepackage[top=1.2in, bottom=1.2in, left=0.6in, right=0.6in]{geometry}

% math packages:
\usepackage{amsmath, amssymb, amsthm, mathtools, bm, moresize, euler}

% graphics packages:
\usepackage{graphicx, epsdice, xcolor, float, wrapfig, caption}


%~~~~~~~~~~~~~~~~~~~~~~~~~~~~~~~~~~~~~~~~~~~~~~~~~~~~~~~~~~~~~~~~~~~~~~~~~~~~~~
%~~~~~~~~~~~~~  0.1. Header Formatting  ~~~~~~~~~~~~~~~~~~~~~~~~~~~~~~~~~~~~~~~

% section header style:
\definecolor{mblu}{rgb}{0.05, 0.40, 0.70}
\newcommand{\msec}[1]{\subsection*{\color{mblu}\textsf{#1}}}

% clear the bibliography's header:
\usepackage{etoolbox}
\patchcmd{\thebibliography}{\section*{\refname}}{}{}{}

%~~~~~~~~~~~~~~~~~~~~~~~~~~~~~~~~~~~~~~~~~~~~~~~~~~~~~~~~~~~~~~~~~~~~~~~~~~~~~~
%~~~~~~~~~~~~~  0.2. Math Symbols and Blocks  ~~~~~~~~~~~~~~~~~~~~~~~~~~~~~~~~~

% fancy letters:
\newcommand{\PP}{\mathbb{P}}
% ...
\newcommand{\Dd}{\mathcal{D}}
\newcommand{\Hh}{\mathcal{H}}
\newcommand{\Ll}{\mathcal{L}}
\newcommand{\Ss}{\mathcal{S}}
\newcommand{\Xx}{\mathcal{X}}

% losses averaged in various ways: 
\newcommand{\Ein}  {\text{trn}_{\Ss}}
\newcommand{\Einb} {\text{trn}_{\check\Ss}}
\newcommand{\Einc} {\text{trn}_{\Ss\sqcup \check\Ss}}
\newcommand{\Egap} {\text{gap}_{\Ss}}
\newcommand{\Eout} {\text{tst}}

% math environment blocks: 
\newtheorem*{qst}{Question}
\newtheorem*{thm}{Theorem}
\newtheorem*{lem}{Lemma}
% ...
\theoremstyle{definition}
\newtheorem*{dfn}{Definition}

%~~~~~~~~~~~~~~~~~~~~~~~~~~~~~~~~~~~~~~~~~~~~~~~~~~~~~~~~~~~~~~~~~~~~~~~~~~~~~~
%~~~~~~~~~~~~~  0.3. Title  ~~~~~~~~~~~~~~~~~~~~~~~~~~~~~~~~~~~~~~~~~~~~~~~~~~~

\begin{document}

{
    \centering \Huge \sf \color{mblu} 
    Kalman Filters
}

%==============================================================================
%=====  1.  HMMS  =============================================================
%==============================================================================

    A fundamental example in statistics is the weighted die.  We may extend it
    into the continuous domain, most simply by considering Gaussians; and we
    may extend it toward modelling correlated sequences, most simply enjoying a
    Markov property.  Doing both extensions gives us the Kalman model of
    continuous-valued sequences.  Here we review inference and estimation in
    each of the four models.

    \msec{Weighted Die}

    \msec{Gaussian}
    \msec{Hidden Markov Models}
    \msec{Kalman Filtering}
        Suppose $x_0 = 0 $,  $x_{t+1} \sim U x_t +
        \mathcal{N}(0, Q)$, and $y_t \sim O x_t + \mathcal{N}(0, R)$, for
        $0\leq t<T$, with all $2T$ noise values independent.  Based on
        $U,Q,O,R$ and on a sequence of observations $\mathfrak{y}_\tau = (y_t : 0\leq
        t<\tau)$, we estimate $\mathfrak{y}_\tau$ by $K(\vec y)$, hoping to minimize the
        expectation of $\|\mathfrak{y}_\tau - K(\vec y_\tau)\|^2$.

        Well, writing $\mu_\tau, C_\tau$ for the mean and variance of $x_\tau$
        conditioned on $\mathfrak{y}_\tau$, we have $\mu_0 = 0$, $C_0 = 0$ and
        \begin{align*}
            &C_{\tau+1}^{-1}(z - \mu_{\tau+1}, z - \mu_{\tau+1})
            =\\
            &(UC_\tau U^{-1} + Q)^{-1}(z - U \mu_\tau, z - U\mu_\tau)  
            +\\
            &R^{-1}(z - y_\tau, O z - y_\tau)
            + \text{const}
        \end{align*}
        and
        $$
            C_{tau+1} =  
        $$

    \msec{Estimating Parameters}
    \msec{Example}

%==============================================================================
%=====  6.  BACKMATTER  =======================================================
%==============================================================================

\end{document}


