% author: samuel tenka
% create: 2021-07-16
% change: 2020-11-27
% descrp: latex source
% to use: run `pdflatex entropic-lower-bounds.tex; evince entropic-lower-bounds.tex`

%~~~~~~~~~~~~~~~~~~~~~~~~~~~~~~~~~~~~~~~~~~~~~~~~~~~~~~~~~~~~~~~~~~~~~~~~~~~~~~
%~~~~~  0. Style Settings  ~~~~~~~~~~~~~~~~~~~~~~~~~~~~~~~~~~~~~~~~~~~~~~~~~~~~

%-------------  0.0. packages  ------------------------------------------------

\documentclass[11pt, justified]{tufte-book}
\usepackage{amsmath, amssymb, amsthm, bm, moresize, euler, scalerel} % math
\usepackage{graphicx, xcolor, listings, float, wrapfig, caption}     % graphics
\geometry{
  left           = 1.0in, % left margin
  textwidth      = 5.0in, % main text block
  marginparsep   = 0.2in, % gutter between main text block and margin notes
  marginparwidth = 1.8in, % width of margin notes
}

%-------------  0.1. macros  --------------------------------------------------
\newcommand{\nosp}{{\hspace{-0.067cm}}}
\newcommand{\wrap}[1]{\left(#1\right)}
\newcommand{\wrsp}[1]{\,\,#1\,\,}
\newcommand{\pr}{\prime}
\newcommand{\lsim}{{\,\scaleobj{0.75}\lesssim\,}}
\newcommand{\gsim}{{\,\scaleobj{0.75}\gtrsim\,}}
\newcommand{\lsims}{{\wrsp{\lsim}}}
\newcommand{\gsims}{{\wrsp{\gsim}}}
\newcommand{\sims}{{\wrsp{\sim}}}

\newcommand{\from}[1][{}]{\xleftarrow{#1}}

\newcommand{\EE}{\mathbb{E}}
\newcommand{\NN}{\mathbb{N}}
\newcommand{\PP}{\mathbb{P}}
\newcommand{\RR}{\mathbb{R}}
%
\newcommand{\Aa}{\mathcal{A}}
\newcommand{\Bb}{\mathcal{B}}
\newcommand{\Cc}{\mathcal{C}}
\newcommand{\Ll}{\mathcal{L}}
\newcommand{\Pp}{\mathcal{P}}
\newcommand{\Qq}{\mathcal{Q}}
\newcommand{\Ss}{\mathcal{S}}
\newcommand{\Tt}{\mathcal{T}}
\newcommand{\Vv}{\mathcal{V}}
\newcommand{\Xx}{\mathcal{X}}
\newcommand{\Zz}{\mathcal{Z}}
%
\newcommand{\Task}{\textsc{task}}
\newcommand{\Ask}{\textsc{ask}}
\newcommand{\Next}{\textsc{next}}
\newcommand{\Read}{\textsc{read}}
\newcommand{\Body}{\textsc{body}}
\newcommand{\Push}{\textsc{push}}

\newcommand{\lt}{\ensuremath <}
\newcommand{\gt}{\ensuremath >}

\newtheorem*{qst}{Question}
\newtheorem*{prop}{Proposition}
\theoremstyle{definition}
\newtheorem*{dfn}{Definition}

\definecolor{mblu}{rgb}{0.05, 0.40, 0.70}
\newcommand{\mtit}[1]{\noindent{\Huge\color{mblu}\bf\textsf{#1}}\vspace{1.5cm}}
\newcommand{\msec}[1]{{\let\clearpage\relax\chapter{\color{mblu}\textsf{#1}}}}
\newcommand{\msub}[1]{\vspace{0.3cm}\section{\color{mblu}\textsf{#1}}}
\newcommand{\mpar}[1]{\par\noindent\textbf{#1}.}

%~~~~~~~~~~~~~~~~~~~~~~~~~~~~~~~~~~~~~~~~~~~~~~~~~~~~~~~~~~~~~~~~~~~~~~~~~~~~~~
%~~~~~  1. Introduction  ~~~~~~~~~~~~~~~~~~~~~~~~~~~~~~~~~~~~~~~~~~~~~~~~~~~~~~

\begin{document}

    \mtit{Entropic Lower Bounds for Sorting}

        \begin{flushright}
            \par``\emph{I play K.\ 330 slowly and evenly, as if X-raying a mollusc.} --- Glenn Gould''
            \par--- Sam Tenka
        \end{flushright}


        The $n \lg n$ lower bound on comparison sorts allures me with its
        beauty.  It goes like this: to identify a total order on a size-$n$ set
        is to distinguish between $n!$ possibilities.  Each interesting query
        to $\leq$ has two possible outcomes and hence yields at most one bit of
        information.  A sorting algorithm based only on comparisons thus
        requires at least $\lg(n!) \sim n \lg n$ many queries; merge sort
        realizes this bound.\footnote{
            We consider complexities up to $\sim$, where $f\sim g$ means $\lim
            f/g = 1$.  We thus attend to constants but not to lower order
            terms.  Here, $\lg=\log_2$.
        }

        This note discusses some lower bounds for related problems.

%~~~~~~~~~~~~~~~~~~~~~~~~~~~~~~~~~~~~~~~~~~~~~~~~~~~~~~~~~~~~~~~~~~~~~~~~~~~~~~
%~~~~~  2. Setting  ~~~~~~~~~~~~~~~~~~~~~~~~~~~~~~~~~~~~~~~~~~~~~~~~~~~~~~~~~~~

    \msec{Counting Queries}
      \msub{Decision Trees}
        The concept of \emph{decision trees} abstracts our notion of algorithm
        to a level convenient for information-theoretic reasoning.  An
        algorithm's complexity is then its height as a tree.  We'll rely only
        on an intuitive understanding of such algorithms and their complexity;
        still, to fix terminology we give a formal definition here.  We
        consider the problem of implementing a map $\Task:\Xx\to\Zz$ in terms
        of queries --- labeled by $q\in\Qq$ --- whose meanings are defined by
        $\Ask:\Qq\to\Xx\to\Aa$.  (We'll consider only finite $\Xx,\Qq,\Aa,\Zz$s
        and surjective $\Task$s.)  For a fixed problem $(\Task, \Ask)$, an
        algorithm of complexity $c$ is a pair $(\Next:\Aa^\star\to \Qq,
        \Read:\Aa^{\times c} \to \Zz)$ that is correct on all inputs:
        \begin{align*}
          (\Task(x), x) &= ((\Read \times {\rm id}_{\Xx}) \circ \Body^{\circ c}) \,\, ([\,], x) \\
          \Body \,\, (\ell, x) &= (\Push \,\, \ell \,\, (\Ask \,\, (\Next \,\, \ell) \,\, x), x)
        \end{align*}
        Here, $\Aa^\star = \bigsqcup_n \Aa^{\times n}$ is the type of
        $\Aa$-valued lists; its constructors are $[\,] \,\, : \,\, \Aa^\star$ (the
        empty list) and $\Push \,\, : \,\, \Aa^\star \to \Aa \to \Aa^\star$.

        We're interested in lower bounds on $c$ for various problems.  For
        example, we may model comparison sorting as the
        problem of computing the identity function $\Task = {\rm id}_{\Xx}$ on
        the set $\Xx$ of total orders on a size-$n$ set $\Ss$ by querying
        $\Ask:(\Ss^2 \setminus \text{diagonal}) \to \Xx\to \{\text{less},
        \text{more}\}$.  As another example, the field of \emph{communication
        complexity} studies problems where $\Task: \text{Alice}\times
        \text{Bob} \to 2$ is a joint predicate and $\Ask: (2^{\text{Alice}}
        \sqcup 2^{\text{Bob}}) \to \Xx \to 2$ is the canonical evaluation map.


      \msub{A Basic Bound}
        Recall the $n\lg n$ argument we started with.   We isolate its essence
        as follows.  A probability distribution on $\Xx$ induces (via $\Task$)
        a distribution on $\Zz$ and (for any fixed algorithm of complexity $c$)
        on $\Qq^c, \Aa^c$.  Let's abuse notation by writing $T$ for the evident
        random variable of type $T$; for instance, ${\rm H}(\Zz)$ is the
        Shannon entropy of the random variable $\textsf{z} =
        \Task(\textsf{x})$.  When we choose $\Xx$'s distribution so that
        $\Zz$'s distribution is uniform, the data processing inequality
        immediately gives the following complexity bound:

        \mpar{Lemma (Counting Bound)}
          $c \lg |\Aa| \geq {\rm H}(\Aa^c) \geq {\rm H}(\Zz) = \lg |\Zz|$.

        As is usual, each problem we consider is actually part of a
        natural-number indexed family of problems and thus induces a sequence
        $(c_n: n\in \NN)$ of complexities.  We study these sequences up to the
        preorder $f\lesssim g$ defined by $\lim f/g \leq 1$.  With this in
        mind, we plug in $|\Xx| = |\Zz| = n!$ and $|\Aa| = 2$ to recover the $c
        \gtrsim n\lg n$ bound for comparison-based identification of total
        orders.

        The counting bound gives some interesting results for related
        identification (i.e., $\Task={\rm id}_\Xx$ problems:

        \mpar{Puzzle (Merge)}
          Let's merge two sorted lists of sizes $m, n$ with $1\ll k\ll n$. 
          More precisely, we fix $\Ss = [k] \sqcup [n]$, set $\Xx$ to the set
          of total orders on $\Ss$ that restrict to the standard orders on
          $[k], [n]$, and let $\Ask$ compare distinct pairs in $\Ss^2$.  Show
          that $c \gtrsim n \lg(n/k)$ and that this bound is achieved.  For
          example, if $k \sim n/\lg n$, then $c \sim n \lg \lg n / \lg n$ is
          optimal --- strictly better than the ``zip'' or ``search'' strategies! 
          This win-win prefigures \emph{fractional cascading}.

        \mpar{Puzzle (Ballots)}
          Let's sort potentially tied elements.  So $\Xx$ contains the ballots
          (a.k.a.: total preorders) on $\Ss$ that have $k$ equivalence
          classes.  Here, $\Ask$ maps to $\{\text{less}, \text{tied},
          \text{more}\}$.  The counting bound says $c \gtrsim n\lg k / \lg 3$. 
          Improve this bound to $c \gtrsim n\lg k$ and show that the latter is
          tight.  Hint: a routine transformation gives for any
          complexity-$c^\prime$ algorithm a complexity-$c\leq c^\prime$
          algorithm that for any $x$ makes fewer than $n$ queries answered by
          ``$\text{tied}$''.

      \msub{Examples in Communication} %Beyond Counting
        \mpar{Communicating an Equality}
        \mpar{Communicating a Comparison}

%~~~~~~~~~~~~~~~~~~~~~~~~~~~~~~~~~~~~~~~~~~~~~~~~~~~~~~~~~~~~~~~~~~~~~~~~~~~~~~
%~~~~~  3. Certificate-Based Arguments  ~~~~~~~~~~~~~~~~~~~~~~~~~~~~~~~~~~~~~~~

    \msec{Convex Cohorts}
      \msub{Distinct Elements}
        \mpar{Strictly Orderable}

        \mpar{Counterfeit Coins}
          We have $n$ coins, some of which may be counterfeit.  Not all of the
          coins are counterfeit.  The counterfeit coins weigh $1+\epsilon$
          while the ordinary coins weigh $1$ for $\epsilon < 1/n$.  We have a
          balance that tells us for any two subsets of the $n$ coins which, if
          any, is heavier.  In how few comparisons may we determine whether or
          not there are any counterfeit coins?

          Now $\Xx = 2^n \setminus [n]$ and $\Task \,\, x$ indicates whether or
          not $\Xx$ is empty.  Now $\Qq = \sum_{k} {n\choose
          k}^2$ and $\Ask \,\, (a,b) \,\, x$ indicates whether or not $|x\cap
          a| = |x \cap b|$.
          %
          What sorts of lower bounds on $c$ can we come up with? 

          Note: in this case algorithms are in correspondence with query
          sequences.  Every element must be hit.  In fact, the graph whose
          edges are "participate in the same weighing, not necessarily on the
          same side" must be connected.

          Call a set of coins that all experience the same weighing pattern 
          of left/none/rights a "coinhort".  

      \msub{Range}
        \mpar{Minimum}
        \mpar{Range}



%~~~~~~~~~~~~~~~~~~~~~~~~~~~~~~~~~~~~~~~~~~~~~~~~~~~~~~~~~~~~~~~~~~~~~~~~~~~~~~
%~~~~~  4. Mutual-Information Based Arguments  ~~~~~~~~~~~~~~~~~~~~~~~~~~~~~~~~

    \msec{Constraints of no Consequence}
      \msub{Length}
      \msub{Bits}

      \begin{tabular}{lcccc}
          problem       &       counting        &       lower       &       upper       &       direct      \\
          \hline
          Merge         &                       &                   & $n\lg\lg n/\lg n$ &                   \\
          Ballots       &                       &                   & $n \lg n/2$       &                   \\
          Strictly      &                       &                   & $n \lg n$         &                   \\
          Counterfeit   &                       &                   & $2\lg n/\lg\lg n$ &                   \\
          Minimum       &                       &                   & $n$               &                   \\
          Range         &                       &                   & $3n/2$            &                   \\
          Length        &                       &                   & $(\lg n)^2/2$     &                   \\
          Bits          &                       &                   & $O(n\lg\lg n)$    &                   \\
      \end{tabular}

%%~~~~~~~~~~~~~~~~~~~~~~~~~~~~~~~~~~~~~~~~~~~~~~~~~~~~~~~~~~~~~~~~~~~~~~~~~~~~~~
%%~~~~~  5. Median Lower Bounds  ~~~~~~~~~~~~~~~~~~~~~~~~~~~~~~~~~~~~~~~~~~~~~~~
%
%    \msec{A Keystone Conundrum}
%      \msub{}
%      \msub{}

\end{document}









