\documentclass[12pt, justified]{tufte-book}
\usepackage{amsmath, amssymb, amsthm, bm, moresize}
\usepackage{euler, tikz-cd}

\usepackage{graphicx, epsdice, xcolor, listings, float, wrapfig, caption}
\geometry{
  left=1in, % left margin
  textwidth=5.5in, % main text block
  marginparsep=0.2in, % gutter between main text block and margin notes
  marginparwidth=1.8in % width of margin notes
}

\usepackage{etoolbox}
\patchcmd{\thebibliography}{\section*{\refname}}{}{}{}

\newcommand{\EE}{\mathbb{E}}
\newcommand{\PP}{\mathbb{P}}
\newcommand{\RR}{\mathbb{R}}
\newcommand{\ZZ}{\mathbb{Z}}
\newcommand{\NN}{\mathbb{N}}
%
\newcommand{\Dd}{\mathcal{D}}
\newcommand{\Ee}{\mathcal{E}}
\newcommand{\Ff}{\mathcal{F}}
\newcommand{\Gg}{\mathcal{G}}
\newcommand{\Hh}{\mathcal{H}}
\newcommand{\Ii}{\mathcal{I}}
\newcommand{\Kk}{\mathcal{K}}
\newcommand{\Ll}{\mathcal{L}}
\newcommand{\Mm}{\mathcal{M}}
\newcommand{\Nn}{\mathcal{N}}
\newcommand{\Ss}{\mathcal{S}}
\newcommand{\Tt}{\mathcal{T}}
\newcommand{\Uu}{\mathcal{U}}
\newcommand{\Vv}{\mathcal{V}}
\newcommand{\Xx}{\mathcal{X}}
\newcommand{\Yy}{\mathcal{Y}}

\newcommand{\smallsquare}{\,\text{\scriptsize$\square$}\,}
\newcommand{\colim}{{\rm colim}}

\newtheorem*{qst}{Question}
\newtheorem*{thm}{Theorem}
\newtheorem*{lem}{Lemma}
\newtheorem*{prop}{Proposition}
\newtheorem*{clm}{Claim}
\theoremstyle{definition}
\newtheorem*{dfn}{Definition}

\newcommand{\mtit}[2]{
  \begin{center}
    \LARGE\sc#1\\
    \Large\rm#2
  \end{center}
}

\newcommand{\msec}[1]{
  \begin{center}\parbox{\textwidth}{\begin{center}
    \hrule\vspace{0.25cm}
    \Large\sc#1\vspace{0.25cm}
    \hrule
  \end{center}}\end{center}
}

\newcommand{\mssec}[1]{
  \begin{center}\parbox{8cm}{\begin{center}
    \large\sc#1
  \end{center}}\end{center}
      \vspace{-0.5cm}
}



\newcommand{\mpar}[1]{\vspace{0.5cm}\par\noindent\textsc{#1} --- }
\newcommand{\abstract}[1]{\begin{center}\parbox{10.5cm}{#1}\end{center}}

\begin{document}
  \renewcommand{\cleardoublepage}{}

  \mtit{Bijections of Finite Sets}

    \abstract{%
      We review a small part of basic combinatorics while emphasizing
      conceptual unity.  For us, combinatorics is the study of bijections
      between finite sets.  This may seem silly seeing as every finite set is
      isomorphic to some standard ordinal $[n] \triangleq \{k\in \NN: k<n\}$.
      But then again, every real space is isomorphic to some free space
      $\RR^{\oplus I}$, yet linear algebra does not collapse into triviality.
      In both cases, those isomorphisms are typically \emph{not canonical}, and
      we find that, while finite sets or vectors spaces
      by themselves are boring, the maps between them are quite rich.
    }

  \msec{Natural Structures}

    \mssec{Joyal's Species}

      \mpar{What is a Species?}
        How may we structure elements into trees, into partitions, etc?  We'll
        focus on what remains when we ignore the elements' internal
        structure.  The categorical concept of \emph{naturality} formalizes
        this focus: we deem a concept significant only when it coheres
        under renaming of elements.

        Let $\textsf{Fin}$ be the category of functions between finite sets and
        let $\textsf{FinBij}$ be its core groupoid.  A \emph{combinatorial
        species} is a functor $:\textsf{FinBij}\to\textsf{Fin}$.  For example,
        the inclusion $I:\textsf{FinBij}\to\textsf{Fin}$ is a species.  We'll
        study the category $\textsf{Species} = [\textsf{FinBij}, \textsf{Fin}]$
        of natural maps between species.
        It is an orienting exaggeration to say that combinatorics is the
        study of $\textsf{Species}$.

        Note that post-composition with $I$ gives an eidetic functor
        $
            [\textsf{FinBij}, \textsf{FinBij}]
            \to
            [\textsf{FinBij}, \textsf{Fin}]
        $.
        Thus, we
        may compose species by composing their corresponding endofunctors.  The
        inclusion $I$ is this composition's identity.

        A functor is \emph{eidetic} when it surjects on objects, arrows
        between those objects, and equalities between those arrows.  The three
        conditions separately we call \emph{esophageal}, \emph{full}, and
        \emph{faithful}.  Indeed, it is through the esophagus that one 
        becomes full of objects.

      \mpar{Maps between Species}

      \mpar{Using Finiteness}
        We restrict attention to \emph{finitary} combinations, since their
        study admits pidgeon-hole principles (e.g.\ in the presence of a
        bijection, every injection is a surjection%; multiplication by a
        %non-empty cardinal is cancelable
        ).  Transfinite induction fails to prove such principles in general
        because limits do not preserve those principles.

    \mssec{Beyond Dimension Zero}
      \mpar{Permutations vs Orders}
      \mpar{Yoneda Lemma}
      \mpar{Eidetic Species}% $S_6$'s Exceptional Automorphisms
        Perhaps the best species are the \emph{eidetic} ones.
        It turns out there are exactly four
        eidetic species up to isomorphism!  
        Two of them swap size-$0$ and size-$1$ sets.
        One of the non-swapping ones is $I$;
        the other one, $E$, acts like $I$ except on sets of size $6$.

        Say $S$ has $6$ elements.  Call a size-$3$ subset of $S$ a `face'.
        $\ZZ/2\ZZ$ acts on the faces by set complement.
        %
        A \emph{$q$-structure} on $S$ is a partition of $S$'s $20$ faces into
        two regular classes interchanged by $\ZZ/2\ZZ$.
        (A size-$10$ class of faces is \emph{regular} if each singleton $\{s\}$
        includes into the same number ($5$) of faces and each doublet $\{s,
        t\}$ includes into the same number ($2$) of faces).  We define $E(S)$
        to be the set of $q$-structures on $S$.

        \begin{thm}
          The eidetic species with functor composition form a group canonically
            isomorphic to $(\ZZ/2\ZZ)^2$.
        \end{thm}

    \mssec{Composing Species}
      \mpar{Algebraic Data Types}
        Since the data of a species is the data of an endofunctor on
        $\textsf{FinBij}$, we may compose combinatorial species (we use
        $\smallsquare$ instead of the standard circle).  The identity under
        composition we have already named ($I$).  Moreover, any $k$-ary
        operation on $\textsf{Fin}$ that preserves $\textsf{FinBij}$ induces a
        $k$-ary operation on $\textsf{Species}$.  Thus we have `pointwise'
        (co)product ($+, \times$) and their identities ($0, 1$).

        For example, 
        we denote $I\times F$ by $\star F$ --- this is called `pointing';
        we denote $F\smallsquare (I+1)$ by $F^\prime$ --- this is called `puncturing'.
        Fans of algebraic data types will recognize these operations as dear
        friends.

      \mpar{Partitions}
        Each species $F$ maps canonically to the species ${\rm Part}$ of
        partitions as follows.  For $t\in F(S)$, let $\pi_t$ be the coarsest
        partition of $S$ such that every $\pi_t$-preserving automorphism of $S$
        also preserves $t$.

        This induces a species $\mathring{F}$ that pointwise includes into
        ${\rm Part}$.   

        As an illustration, we present Joyal's calculation:
        \begin{align*}
            {\rm Ord}\times {\rm Endo}
&\cong
            {\rm Ord}\times ({\rm Perm} \circ (\star{\rm Tree}))
\\&\cong        
            {\rm Ord}\times ({\rm Ord} \circ (\star{\rm Tree}))
\\&\cong
            {\rm Ord}\times (\star\star{\rm Tree})
        \end{align*}
        %Only the middle isomorphism uses the ${\rm Ord}$ factor.
        Because this map preserves its first factor, we summarize as follows.
        \begin{thm}
          In the presence of a total order, endomorphisms correspond naturally
          with bipointed trees.
        \end{thm}
        An interesting corollary is Cayley's: there are $n^{n-2}$ trees on $n$
        nodes.
        
        An endofunction's core is the categorical limit of $\cdots\to S\to S$.
        To give a pointed tree is to give an endofunction with size-one core
        (pidgeonhole).

        To give a permutation is to give an endofunction with full-size core.
        %Fix $f:S\to S$.  Say $s\sim f(s)$ when $f(s) 

      \mpar{Atomic Decomposition}
        Each $S_n$-set decomposes canonically as a sum of orbits.  We call such
        orbits \emph{molecules}.  With coproduct and cauchy product, ${\rm
        Species}$ is a (class-sized) commutative semiring with multiplicative
        unit $X_0 = [- , 0]$. 
        A molecule $p$ is \emph{atomic} when it is irreducible with respect 
        to the cauchy product.

        \begin{thm}
            Moo
        \end{thm}

        There are $0, 1, 1, 2, 6, 6, 27, 20, \cdots$ many atoms on 
        $0, 1, 2, 3, 4, 5, 6, 7, \cdots$ elements.

      %\mpar{Atomic Decomposition}
        

    \mssec{Making Choices}
      \mpar{Products don't Reflect Isomorphisms}
      \mpar{Finitary Choice Principles}
        A size-$n$ choice principle on a set $S$ is a simultaneous pointing for
        each of $S$'s size-$n$ subsets.  Let the species ${\rm Ch}_n$ give a set's
        size-$n$ choice principles.
        $$
            {\rm Ch}_n = E \smallsquare {\rm Subs}_n
        $$
        For example, we may visualize $C_2$ as
        sending $S$ to the set of complete directed graphs on $S$.

        What data can witness a choice principle?  That is, which species map
        into $C_n$?

        Suppose we have a way to choose an element from each $k$-element
        set.  Do we then have a canonical way to choose an element from each
        $K$ element set $S$?  

        For example, say $k=2$ and $K=4$.  The following method works. 
        Take the set $P$ of two-part partitions of $S$ into size-two subsets.
        For each of $P$'s three elements, apply the choice function to select
        an element of $S$.  The image is a multi-set $M$ of three elements in
        $S$.  If the elements are all distinct, let us output the one missing
        element.  Otherwise, there is an element that occurs strictly more
        times in $M$ than any other element occurs; return that element.

        More precisely, we have demonstrated a natural map between
        species:
        $$
            {\rm Ch}_2 \times ({\rm Ch}_2 \smallsquare {\rm Subs}_2) \to [{\rm Part}_{2+2},I] \to {\rm Ch}_4
        $$
        Here, ${\rm Subs}_2$ gives a set's size-$2$ subsets; ${\rm Part}_{2+2}$
        gives a set's partitions of type $2+2$; and $[,]$ is the pointwise
        exponential.

  \msec{Abstraction and Symmetry}
    \mssec{}
    \mssec{Poly\'a Counting}
    \mssec{}

  \msec{Approximation and Order}
    \mssec{}
    \mssec{M\"obius Inversion}
    \mssec{}

  %\msec{Averages and Eigenmodes}
  %  \mssec{}
  %  \mssec{}
  %  \mssec{}
  %  \mssec{}
  %  \mpar{Encoding Septuples of Trees}

\end{document}


