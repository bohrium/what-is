\documentclass[justified]{tufte-book}
\usepackage{amsmath, mathtools, amssymb, pifont}
\usepackage{graphicx, xcolor, listings}
\usepackage{tikz-cd}

\newcommand{\cmark}{\ding{51}}%
\newcommand{\xmark}{\ding{55}}%

\usepackage{wrapfig, tikz-cd, subcaption, multicol, adjustbox, cancel, graphics}
\renewcommand*{\thefootnote}{\color{red}\fnsymbol{footnote}}
\newcommand{\incltik}[1]{\scalebox{1.0}{$\begin{array}{l}#1\end{array}$}}
\newcommand{\jetsquare}{
{\begin{tikzcd}
JE \arrow[d] \arrow[r] & TE\otimes T^*B \arrow[d] \\
B \arrow[r]            & TB\otimes T^*B          
\end{tikzcd}}
}

\makeatletter
\newcommand*\bigcdot{\mathpalette\bigcdot@{.5}}
\newcommand*\bigcdot@[2]{\mathbin{\vcenter{\hbox{\scalebox{#2}{$\m@th#1\bullet$}}}}}
\makeatother

\begin{document}

  \title{Two Kinds of Hole}
  \maketitle

  \setcounter{tocdepth}{1}
  \tableofcontents
  \newpage
  
  \chapter{Dimensions $0$ and $1$}
    \section{Homotopy and Cohomology}
      When is a nice space --- say, a manifold $X$ --- connected?
      On one hand, $X$ is connected when any two points in $X$
      may be joined by a path; this has to do with maps \emph{to} $X$.
      On the other, $X$ is connected when each locally constant
      function is globally constant; this has to do with maps \emph{from} $X$.
      We formalize these two notions as follows.
      We work with based spaces.
      The $k$th \emph{rational homotopy} of $X$ is the
      set of formal $\mathbb{Q}$-linear combinations of maps $f:S^k\to X$, 
      mod homotopies between $f,g:S^k\to X$, and mod addition defined below.
      The $k$th \emph{rational cohomology} of $X$ is the set of formal 
      $\mathbb{Q}$-linear combinations of maps $f:X\to S^k$,
      mod homotopies between $f,g:X\to SP(S^k)$, and mod addition defined below.
      TODO: are $\mathbb{Q}$-linear combinations of maps $f:X\to S^k$ the same as
      maps from $f:X\to SP(S^k)$ up to homotopy?
      (Perhaps assume compactness of $X$ or think only about compactly supported
      cohomology?)
      
    \section{Addition}
    \section{Multiplication}
    \section{Graphs and Covers}
      We might ask for spaces with vanishing $2$nd and higher homotopy
      and cohomology (not just rational).  Examples are graphs-with-higher-cells
      Or we might ask for spaces with vanishing $1$st and lower homotopy
      and cohomology (not just rational).  These are the simply connected
      spaces.  For any $X$ we have good maps $C\to X\to G$ from a simply
      connected space to $X$ to a graph-with-higher-cells.
    
  \chapter{Dimensions $2$ and higher}
    \section{Differential forms}
    \section{Poincare duality}
    \section{Sullivan models}
    \section{Sample computations}
  \chapter{Applications}
    \section{Extension and lifting}
    \section{Vector bundles}
    \section{Fixed points}
    \section{Intersections}
    
  
\end{document}


