% author:   sam tenka
% create:   2022-04
% change:   2022-05-02

%==============================================================================
%=====  0.  DOCUMENT SETTINGS  ================================================
%==============================================================================

%~~~~~~~~~~~~~~~~~~~~~~~~~~~~~~~~~~~~~~~~~~~~~~~~~~~~~~~~~~~~~~~~~~~~~~~~~~~~~~
%~~~~~~~~~~~~~  0.0. About and Beyond this Exposition  ~~~~~~~~~~~~~~~~~~~~~~~~

%---------------------  0.0.0. page geometry  ---------------------------------
\documentclass[11pt]{article}
\usepackage[top=3cm, bottom=3cm, left=3cm, right=3cm]{geometry}

%---------------------  0.0.1. math packages  ---------------------------------
\usepackage{amsmath, amssymb, amsthm, mathtools, bm, moresize, relsize}%, euler}

%---------------------  0.0.2. graphics packages  -----------------------------
\usepackage{graphicx, epsdice, xcolor, float, wrapfig, caption, hyperref}

%~~~~~~~~~~~~~~~~~~~~~~~~~~~~~~~~~~~~~~~~~~~~~~~~~~~~~~~~~~~~~~~~~~~~~~~~~~~~~~
%~~~~~~~~~~~~~  0.1. Header Formatting  ~~~~~~~~~~~~~~~~~~~~~~~~~~~~~~~~~~~~~~~

\definecolor{mred}{rgb}{0.90, 0.05, 0.05}
\newcommand{\red}{\color{mred}}

\definecolor{mblu}{rgb}{0.05, 0.35, 0.70}
\newcommand{\blu}{\color{mblu}}

\definecolor{mbre}{rgb}{0.30, 0.45, 0.60}
\newcommand{\bre}{\color{mbre}}

\definecolor{mgre}{rgb}{0.55, 0.55, 0.50}
\newcommand{\gre}{\color{mgre}}

%---------------------  0.1.0. tidbit headers  --------------------------------
\newcommand{\samtitle} [1]{
  \par\noindent{\Huge \sf \blu #1}
  \vspace{0.1cm}
}

\newcommand{\samquote} [2]{
    \par\noindent{\begin{flushright}\parbox{10cm}{
    \begin{flushright}
    \normalsize \gre {\it #1} \\ --- #2
    \end{flushright}
    }\end{flushright}}
}

%---------------------  0.1.1. section headers  -------------------------------

\newcommand{\samsection} [1]{
  \vspace{0.2cm}
  \par\noindent{\LARGE \sf \blu #1}
  \vspace{0.1cm}\par
}

\newcommand{\samsubsection}[1]{
  \vspace{0.2cm}
  \par\noindent{\Large \sf \bre #1}
  \vspace{0.1cm}\par
}

\newcommand{\samsubsubsection}[1]{
   \vspace{0.1cm}
   \par\noindent{\hspace{-2cm}\normalsize \sc \gre #1} ---
}

\newcommand{\note}[1]{{\blu \textsf{#1}}}
\newcommand{\attn}[1]{{\red \textsf{#1}}}

%---------------------  0.1.2. clear the bibliography's header  ---------------
\usepackage{etoolbox}
\patchcmd{\thebibliography}{\section*{\refname}}{}{}{}

%~~~~~~~~~~~~~~~~~~~~~~~~~~~~~~~~~~~~~~~~~~~~~~~~~~~~~~~~~~~~~~~~~~~~~~~~~~~~~~
%~~~~~~~~~~~~~  0.2. Math Symbols and Blocks  ~~~~~~~~~~~~~~~~~~~~~~~~~~~~~~~~~

%---------------------  0.2.0. math markers, delimiters, operators  -----------

\newcommand{\pr}{\prime}

\newcommand{\scirc}{\mathrel{\mathsmaller{\mathsmaller{\mathsmaller{\circ}}}}}
\newcommand{\cmop}[2]{{(#1\!\to\!#2)}}

%---------------------  0.2.1. math helpers  ----------------------------------

%---------------------  0.2.2. probability symbols  ---------------------------

\newcommand{\KL}{\text{KL}}
\newcommand{\EN}{\text{H}}
\newcommand{\MI}{\text{MI}}

%---------------------  0.2.3. double-struck and caligraphic letters  ---------

\newcommand{\Aa}{\mathbb{A}}\newcommand{\aA}{\mathcal{A}}
\newcommand{\Bb}{\mathbb{B}}\newcommand{\bB}{\mathcal{B}}
\newcommand{\Cc}{\mathbb{C}}\newcommand{\cC}{\mathcal{C}}
\newcommand{\Dd}{\mathbb{D}}\newcommand{\dD}{\mathcal{D}}
\newcommand{\Ee}{\mathbb{E}}\newcommand{\eE}{\mathcal{E}}
\newcommand{\Ff}{\mathbb{F}}\newcommand{\fF}{\mathcal{F}}
\newcommand{\Gg}{\mathbb{G}}\newcommand{\gG}{\mathcal{G}}
\newcommand{\Hh}{\mathbb{H}}\newcommand{\hH}{\mathcal{H}}
\newcommand{\Ii}{\mathbb{I}}\newcommand{\iI}{\mathcal{I}}
\newcommand{\Jj}{\mathbb{J}}\newcommand{\jJ}{\mathcal{J}}
\newcommand{\Kk}{\mathbb{K}}\newcommand{\kK}{\mathcal{K}}
\newcommand{\Ll}{\mathbb{L}}\newcommand{\lL}{\mathcal{L}}
\newcommand{\Mm}{\mathbb{M}}\newcommand{\mM}{\mathcal{M}}
\newcommand{\Nn}{\mathbb{N}}\newcommand{\nN}{\mathcal{N}}
\newcommand{\Oo}{\mathbb{O}}\newcommand{\oO}{\mathcal{O}}
\newcommand{\Pp}{\mathbb{P}}\newcommand{\pP}{\mathcal{P}}
\newcommand{\Qq}{\mathbb{Q}}\newcommand{\qQ}{\mathcal{Q}}
\newcommand{\Rr}{\mathbb{R}}\newcommand{\rR}{\mathcal{R}}
\newcommand{\Ss}{\mathbb{S}}\newcommand{\sS}{\mathcal{S}}
\newcommand{\Tt}{\mathbb{T}}\newcommand{\tT}{\mathcal{T}}
\newcommand{\Uu}{\mathbb{U}}\newcommand{\uU}{\mathcal{U}}
\newcommand{\Vv}{\mathbb{V}}\newcommand{\vV}{\mathcal{V}}
\newcommand{\Ww}{\mathbb{W}}\newcommand{\wW}{\mathcal{W}}
\newcommand{\Xx}{\mathbb{X}}\newcommand{\xX}{\mathcal{X}}
\newcommand{\Yy}{\mathbb{Y}}\newcommand{\yY}{\mathcal{Y}}
\newcommand{\Zz}{\mathbb{Z}}\newcommand{\zZ}{\mathcal{Z}}

%---------------------  0.2.2. math environments  -----------------------------
\newtheorem*{qst}{Question}
\newtheorem*{thm}{Theorem}
\newtheorem*{lem}{Lemma}
% ...
\theoremstyle{definition}
\newtheorem*{dfn}{Definition}

%~~~~~~~~~~~~~~~~~~~~~~~~~~~~~~~~~~~~~~~~~~~~~~~~~~~~~~~~~~~~~~~~~~~~~~~~~~~~~~
%~~~~~~~~~~~~~  0.3. Section Headers  ~~~~~~~~~~~~~~~~~~~~~~~~~~~~~~~~~~~~~~~~~


\begin{document}
\samtitle{Diagonalization of Dependencies}

  \samsection{Towards Trivariate Mutual Information}
    \samsubsection{Challenges in Generalizing $\MI$}
      \samsubsubsection{Inclusion-Exclusion}
      \samsubsubsection{Venn Diagrams Mislead}
        My intuition is that mutual information in $3$ variables is so slippery
        to define-with-complete-consensus due to a failure of
        additivity-of-information.  An analogous situation occurs in the
        dimension theory of vector spaces: Note that if $A,B,C$ are subspaces
        of a finite dimensional vector space, then
        $$
          \dim(A+B) = \dim(A) + \dim(B) - \dim(A \cap B)
        $$
        $$
          \dim(A\cap B) = \dim(A) + \dim(B) - \dim(A+B)
        $$
        These look like inclusion-exclusion in a venn diagram.  Are the
        following then true?
        $$
          \dim(A+B+C) = \dim(A) + \dim(B) + \dim(C)
                      - \dim(A \cap B) - \dim(B \cap C) - \dim(C \cap A)
                      + \dim(A \cap B \cap C) 
        $$
        $$
          \dim(A\cap B\cap C) = \dim(A) + \dim(B) + \dim(C)
                              - \dim(A + B) - \dim(B + C) - \dim(C + A)
                              + \dim(A + B + C) 
        $$
        No!  That's what we would expect from a venn diagram, but it is wrong.
        Likewise, venn models $\MI$ badly.  For these reasons, I think
        $\MI(X;Y;Z)$ is not a real number but instead something that remembers
        more geometric structure (much as a lattice of subspaces generated by
        sums and intersections of given subspaces carries more data than
        just the dimensions of the pure sums).

        Fix a vector space with basis $u,v,x$.
        Let subspaces $A,B,C$ be spanned by $\{u\},\{v\},\{u+v\}$.
        Let subspaces $A^\pr,B^\pr,C^\pr$ be spanned by $\{u\},\{v\},\{x\}$.
        Each of these spaces has dimension $1$.
        The pairwise and triple intersections within each triplet all have dimension $0$.
        And yet the two situations are not isomorphic, since $C\subseteq A+B$
        but $C^\pr\subsetneq A^\pr+B^\pr$.  Another example is that $(A,B,C)$
        have the same intersection dimensions as $(B,C,A)$, even though $C$ is
        distinguished.
      \samsubsubsection{Distributivity}

      \samsubsubsection{Geometry}

    \samsubsection{Intersections, Complements, Cohomology}

    \samsubsection{Geometry}
      

\vspace{0.5cm}
\hrule
\vspace{0.1cm}
\hrule
\vspace{0.1cm}

    \begin{verbatim}
            x1'    x2'    y'
             |     |      |
            x1-----x2-----y
                   |
                  x2'
    \end{verbatim}
    We're studying a perturbation of a system exactly described by a markov
    structure (shoe horizontal edges far from nearly-independent).  One
    interesting task in this setting is to predict y based on x2, then
    fine-tune based on x1.  This relates to self-supervised learning for
    representation learning; pretraining.  
    %
    We want to perturb in ``p(y|x2) vs p(y|x2,x1)''; perhaps the right analogue
    to the original story is for us to assume access to two-point but not
    three-point marginals.

    Start with understanding for one simple SVD in nearly markov how to
    interpret that SVD.

    Now, \emph{what are analogues of the original story's singular vectors?}
    %
    The bounded-computation framework suggests that we maximize something like
    $$
        \MI^\pr(x1^\pr;y^\pr \,|\, x2)
        \quad
        \quad
        \text{or maybe}
        \quad
        \quad
        \MI^\pr(x1^\pr;y^\pr \,|\, x2^\pr)
    $$
    The thing to do, though, is to define $\MI^\pr$!
    For instance, might we condition on a ``mediating'' variable just by taking
    an expectation over the $\MI^\pr$s of its conditionals, as follows?
    $$
      \MI^\pr(X;Z\,|\,M) = \Ee_{m\sim \mu_{M}} \sup_{c~\text{bilinear}}\,
      \Ee_{x,z \sim \mu_{X\times Z|M}(\cdot|m)} \left[
        c(x,z)
      \right]
      -
      \log
      \Ee_{x,z \sim \mu_{X|M}(\cdot|m) \times \mu_{Z|M}(\cdot|m)} \left[
      \exp
        c(x,z)
      \right]
    $$
    Or should $c$ instead somehow know about $m$ as well?

    Relatedly, if we condition on $x2^\pr$ then is there a way we would
    like to incorporate $x2^\pr$'s linear structure?
    And if we condition instead on $x2$, is there a decomposition theorem reducing to old work.


\vspace{0.1cm}
\hrule
\vspace{0.1cm}
\hrule
\vspace{0.5cm}


  In deep learning, we implicitly characterize a concept-to-be-learned by
  specifying how it should behave in composition with other concepts.  That is,
  \emph{we build our tools by using them}.  This Grothendieckian framework
  sheds light on many global architectural themes of deep learning:
  autoencoders, cycle consistency losses, attention layers, , siamese networks,
  contrastive losses.  A key example is self-supervised representation
  learning, wherein we learn to represent data by attempting to predict some
  of its ``parts'' or ``aspects'' from others. 

  In these notes we elaborate on this theme in the case of weak dependencies.
  %methods of representating data based on such
  %inter-part dependency information.

  %\note{What is a revealing way to visualize or plot a joint distribution on
  %two nearly independent categorical variables?}
  In what sense is the left figure a better featurization of our data than the right? 

  \samsection{Universal Features}
    \samquote{
      He had bought a large map representing the sea, \\
      Without the least vestige of land: \\
      And the crew were much pleased when they found it to be \\
      A map they could all understand.
    }{The Hunting of the Snark}


    We summarize the (\href{https://arxiv.org/abs/1911.09105}{2019 theory}) of
    Shao-Lun Huang, Anuran Makur, Gregory W.\ Wornell, and Lizhong Zheng. 

    \samsubsection{MI between nearly joint-uniform variables}
      Suppose $X,Z$ have uniform marginals and are nearly independent --- 
      $$
        p(x^\pr, z^\pr) = \frac{\exp(\epsilon \pi(x^\pr, z^\pr))}{|X\times Z|} 
      $$
      --- in that their mutual information $\MI(X;Z)$ is $\ll \epsilon^2$.
      Here, $\pi(\cdot, \cdot) \ll 1$.
      Write $M=|X\times Z|$.

      \samsubsubsection{Bird's Eye Story}
        We wish to featurize $X,Z$ into
        $k$-dimensional variables $X^\pr, Z^\pr$ so as to preserve as much shared
        information as possible:
        $$
          X^\pr - X - Z - Z^\pr
          \quad\quad
          \MI^\pr(X^\pr;Z^\pr) \lesssim \MI(X;Z)
        $$

        Now, our notion $\MI^\pr$ of shared information ought not literally be
        mutual information; otherwise, for positive $k$ almost all embeddings of
        $X,Z$ into $\Rr^k$ would trivially preserve information. 
        %
        Still, there is a sense in which some embeddings of $X,Z$ are ``more
        aligned'' or ``more co-informative'' than others.  We'll focus on the
        notion of \emph{accessible co-informativity}: that is, on a measure of
        how related variables appear despite bounded computation to compare them.
        %
        Variational bounds provide a convenient way to model bounded computation:
        $$
          \MI(X,Z) = \sup_c\,
          \Ee_{x,z \sim \mu_{X\times Z}} \left[
            c(x,z)
          \right]
          -
          \log
          \Ee_{x,z \sim \mu_{X} \times \mu_{Z}} \left[
          \exp
            c(x,z)
          \right]
        $$
        with the supremum achieved at $c_{x,z}=\log(p_{x,z}/(p_x p_z))=\epsilon \pi_{x,z}$.
        Let's define $\MI^\pr$ to be the same thing except the function $c$ over
        which we vary must be \textbf{bilinear}.
 
        What does that $\MI$ objective look for $c$ near the unconstrained optimum
        $c=\epsilon \pi$?  Well, for $c=\epsilon(\gamma+\pi)$
        we have to second order in $\epsilon$ and terms constant in $\gamma$:
        \begin{align*}
          \cdots
          &=
          \sum_{x,z} \epsilon\gamma_{x,z} \cdot \exp(\epsilon\pi_{x,z})/M
          -
          \log\left(1 + \sum_{x,z} (\exp(\epsilon \gamma_{x,z})-1) \cdot \exp(\epsilon\pi_{x,z})/M\right)
          \\
          &\in
          C-\frac{1}{2}
          \sum_{x,z} (\epsilon\gamma_{x,z})^2 \cdot \exp(\epsilon\pi_{x,z})/M
          +
          \frac{1}{2}
          \left(\sum_{x,z} \epsilon \gamma_{x,z} \cdot \exp(\epsilon\pi_{x,z})/M\right)^2
          + o(\epsilon^2)
          \\
          &=
          C-\frac{\epsilon^2}{2}
            \text{Var}_{x,z\sim \mu_{X\times Z}} [\gamma_{x,z}] + o(\epsilon^2)
        \end{align*}
        Neither the true nor approximate $\MI$ objectives care whether we
        translate $\gamma$ vertically.  So let's take $\gamma$ to have mean zero
        with respect to the true joint.
        %
        The above says that for constrained $c$ (so long as the feasible set
        gets very close to the true optimum), we want a constrained $c$ that
        minimizes a least-squares error 
        $\text{Var}_{x,z\sim \mu_{X\times Z}} [\epsilon\pi - c]$ --- wow!

        Let's consider $c$ bilinear in $x^\pr,z^\pr$ (the latter featurized per $f,g$):
        $$
          c_{x,z} = \epsilon \sum_k f_x^k g_z^k 
        $$
        Momentarily neglecting centering and the difference between variance with
        respect to joint vs independent marginals, we see that the SVD minimizes
        that least-squares error!
        \par\noindent
        \textbf{Desired Result}: \emph{To leading order in $\epsilon$, the
        optimal features $f,g$ are proportional to the top $k$ non-trivial
        singular vectors of $\exp(\epsilon\pi)$ (when SVD and optimization are
        uniquely soluble).} 

      \samsubsubsection{Technical Details}
        %\attn{We should be more careful, so we state
        %the following conjecturally (TODO: also address that perturbative
        %analysis of ill-conditioned and/or constrained optimization must be
        %handled delicately)}
        
        This result justifies the truncation developed in HMWZ 2019 (\S IV.C).

      \samsubsubsection{Comparison to HMWZ}
        That paper characterizes its truncation as a solution to five
        optimization problems (\S III.B; \S V.C thru \S V.F), each of which :
        (\attn{TODO: verify!})
        \begin{table}
          \begin{tabular}{c|cccccc}
                        &\S III.B       &\S V.C         &\S V.D         &\S V.E         &\S V.F         &   ours        \\ \hline
            norm        &   Ky-fan      &               &               &               & nuclear       & information   \\
            constraint  &               &               &               & feature noise &               & computation   \\
                        &   Ky-fan      &               &               &               &               &                 
          \end{tabular} 
        \end{table}
        \begin{itemize}
          \item relies on a norm introduced manually rather than derived from $\KL$ (Ky-fan, nuclear, ) or
          \item supposes, besides near-independence of $X,Z$, an asymptotically extreme feature noise
        \end{itemize}
        By contrast, in our analysis the universal features fall out from a
        linearity constraint (having to do with pure linearity rather than
        norms).

      % TODO: compare to max correlation analysis in section 3
      %max f, g \Ee(f(x) g(y))
      % TODO: what questions does this bounded-computation viewpoint help answer? 
      % --- analyze random features idea?
      % TODO: 
      % 

      %Then, very explicitly, we have
      %\begin{align*}
      %  \MI^\pr
      %  &=
      %  \sup_{f,g}
      %  \epsilon \sum_{x,z} \exp(\epsilon\pi_{x,z}) \sum_k f_x^k g_z^k/|X\times Z|
      %  -
      %  \log(\sum_{x,z} \exp(\epsilon \sum_k f_x^k g_z^k)/|X\times Z|)
      %  \\
      %  &\approx
      %  \sup_{f,g}
      %  \epsilon \sum_{x,z} \exp(\epsilon\pi_{x,z}) \sum_k f_x^k g_z^k/|X\times Z|
      %  -
      %  \log(\sum_{x,z} \exp(\epsilon \sum_k f_x^k g_z^k)/|X\times Z|)
      %\end{align*}
      %(Use compactness+continuity to pull $O(\epsilon^d)$ out of the $\sup$).

    \samsubsection{adaptation to non-uniform marginals}
    \samsubsection{principal components}
      We review principal component analysis.
      %\samsubsubsection{pythagoreas, grassmann, gauss}
      \samsubsubsection{pythagoreas and gauss} 
      \samsubsubsection{sidetrack: causal grassmanians}
        Pearl's do-calculus --> fit a nilpotent matrix to data?!
        % A--B--C--D--E--F
        PCA:correlation :: ?:causation

        \begin{verbatim}
        (I + T) nxn
        T^n = 0 
         FED
        (1... 
        ( 1..
        (  1.
        (   1)
        \end{verbatim}

        \attn{TODO: in nearly markov case, notion of ``orthogonality'' is
        different... offdiagonals ... perhaps a triangularization?? what does this look like?}

      \samsubsubsection{}
      \samsubsubsection{fischer metric}

    \samsubsection{greg's analysis of weak pairwise dependencies}

  \samsection{Nearly Markov}
    \samquote{
      Independence is a heady draft...
    }{Maya Angelou}

    \begin{verbatim}
            x1'      y'
             |       |
            x1 - x2- y
                 |
                x2'
    \end{verbatim}
    In nearly markov structure, predict y based on x2;
    and fine-tune based on x1 --- relate to self-supervised learning
    for representation learning; pretrainig.  (Assume have two-point but not three-point
    marginals).
    %
    \attn{PRIORIZZE THIS PARAGRAPH perturb in ``p(y|x2) vs p(y|x2,x1)''}
    %
    --- what is analogue of singular vectors/values here?
    % ALSO CONSIDER:  

  \samsection{Generalized Self-Supervised}
  \samsection{Sample Complexity}



\end{document}


      %$c_{x,z}-c_{\tilde x, \tilde z} \in O(\epsilon)$ as well.  Say $c_{x,z} =
      %\epsilon \gamma_{x,z}$.  Then to first order in $\epsilon$:
      %\begin{align*}
      %  1 + \epsilon \pi_{x,z}
      %  \approx
      %  (1 + \epsilon \gamma_{x,z})
      %  \frac{|X\times Z|}
      %    {\sum 1 + \epsilon \gamma_{\tilde x, \tilde z}}
      %\end{align*}
      %and $\gamma \approx \pi$ (remember that $\sum \pi = 0$ exactly). 
      %We can expand to second order to see deviations from proportionality:
      %\begin{align*}
      %  \exp(\epsilon \pi_{x,z})
      %  \approx
      %  \exp(\epsilon \gamma_{x,z})
      %  \frac{|X\times Z|}
      %    {\sum 1 + \epsilon \gamma_{\tilde x,\tilde z} + \epsilon^2 \gamma_{\tilde x,\tilde z}^2/2}
      %\end{align*}


%      The $\MI$ objective's partial with respect to $c_{x,z}$ is:
%      \begin{align*}
%        p_{x,z}
%        -
%        p_{x} p_{z} \cdot
%        \frac{\exp c_{x,z}}{\sum p_{\tilde x} p_{\tilde z}
%        \exp c_{\tilde x,\tilde z}}
%        = 0
%      \end{align*}
%      Using $p_{x,z} / \exp(\epsilon \pi_{x,z}) = p_{x} p_{z} = 1/|X\times Z|$, we have
%      \begin{align*}
%        \exp(\epsilon \pi_{x,z})
%        =
%        \exp(\epsilon \gamma_{x,z})
%        \cdot
%        \frac{|X\times Z|}
%          {\sum 1 + \epsilon \gamma_{\tilde x,\tilde z} + \epsilon^2 \gamma_{\tilde x,\tilde z}^2/2 + \cdots}
%      \end{align*}
%      Thus, $\exp(\epsilon \pi_{x,z})$ and $\exp(\epsilon \gamma_{x,z})$
%      are proportional and 
%      {\color{red}
%      (TODO: address that ill-conditioned constrained optimization
%      can be untreatable by perturbation)} 
%      $$
%        \epsilon\gamma = \epsilon\pi +  
%        \log\left(1 + \sum \frac{\exp(\epsilon\gamma_{\tilde x,\tilde z})-1}{|X\times Z|}\right)
%      $$
%      Now, assuming $\gamma - \pi \in O(\epsilon)$, that $\log$ term is
%      $O(\epsilon^2)$ since $\sum \exp(\epsilon\pi)-1 = 0$ exactly.
%      %
%      This justifies a perturbative expansion:
%      $$
%        \gamma = \pi +
%        \epsilon
%        \sum \frac{\exp(\epsilon\pi_{\tilde x,\tilde z})-1}{|X\times Z|} + o(\epsilon) 
%      $$
%
%
%      Now, the objective's partial with respect to $f_x^k$ is:
%      $$
%        \epsilon \lambda_k = 
%        \frac{\epsilon \sum_{\tilde z} g_{\tilde z}^k \exp(\epsilon \pi_{x,\tilde z})}
%        {\sum_{\tilde x, \tilde z} 1} 
%        -
%        \frac{\epsilon \sum_{\tilde z} g_{\tilde z}^k \exp\left(\epsilon \sum_{\tilde k} f_{x}^{\tilde k} g_{\tilde z}^{\tilde k}\right)}
%        {\sum_{\tilde x, \tilde z} \exp\left(\epsilon \sum_{\tilde k} f_{\tilde x}^{\tilde k} g_{\tilde z}^{\tilde k}\right)}
%      $$
%      We introduced a lagrange multiplier since
%      we insist on the constraint that $\sum_x f_x^k = 0 = \sum_z g_z^k$,
%      i.e., that the embeddings are centered.  This makes both denominators 
%      above $1+O(\epsilon^2)$.  To leading order:
%      $$
%        \lambda_k = \epsilon \sum_{\tilde z} g_{\tilde z}^k (\pi_{x,\tilde z} - \sum_{\tilde k} f_{x}^{\tilde k} g_{\tilde z}^{\tilde k})
%      $$
%      We SUPPOSE THAT $\lambda=0$; then for each $x$, $g^k$ is orthogonal
%      to the residual $\pi - c$.
%      {\color{red}
%
%
