\documentclass[11pt, justified]{tufte-book}
\usepackage{amsmath, amssymb, amsthm, bm, moresize}
\usepackage{euler, tikz-cd}

\usepackage{graphicx, epsdice, xcolor, listings, float, wrapfig, caption}
\geometry{
  left=1in, % left margin
  textwidth=5.0in, % main text block
  marginparsep=0.2in, % gutter between main text block and margin notes
  marginparwidth=1.8in % width of margin notes
}

\usepackage{etoolbox}
\patchcmd{\thebibliography}{\section*{\refname}}{}{}{}

\newcommand{\EE}{\mathbb{E}}
\newcommand{\PP}{\mathbb{P}}
\newcommand{\RR}{\mathbb{R}}
\newcommand{\ZZ}{\mathbb{Z}}
%
\newcommand{\Dd}{\mathcal{D}}
\newcommand{\Ee}{\mathcal{E}}
\newcommand{\Ff}{\mathcal{F}}
\newcommand{\Gg}{\mathcal{G}}
\newcommand{\Hh}{\mathcal{H}}
\newcommand{\Ii}{\mathcal{I}}
\newcommand{\Kk}{\mathcal{K}}
\newcommand{\Ll}{\mathcal{L}}
\newcommand{\Mm}{\mathcal{M}}
\newcommand{\Nn}{\mathcal{N}}
\newcommand{\Ss}{\mathcal{S}}
\newcommand{\Tt}{\mathcal{T}}
\newcommand{\Uu}{\mathcal{U}}
\newcommand{\Vv}{\mathcal{V}}
\newcommand{\Xx}{\mathcal{X}}
\newcommand{\Yy}{\mathcal{Y}}

\newtheorem*{qst}{Question}
\newtheorem*{thm}{Theorem}
\newtheorem*{lem}{Lemma}
\newtheorem*{prop}{Proposition}
\newtheorem*{clm}{Claim}
\theoremstyle{definition}
\newtheorem*{dfn}{Definition}

\definecolor{mblu}{rgb}{0.05, 0.40, 0.70}
\newcommand{\mtit}[1]{%
    {%
        \noindent
        \Huge\color{mblu}\bf\textsf{#1}
        \vspace{0.3cm}
    }
}
\newcommand{\msec}[1]{\chapter{\color{mblu}\textsf{#1}}}
\newcommand{\mssec}[1]{\section{\color{mblu}\textsf{#1}}}
\newcommand{\mpar}[1]{\par\textsc{#1} --- }

\begin{document}
  \renewcommand{\cleardoublepage}{}

  \mtit{Small Oscillations in Finite Dimension}

  \msec{Modes}
    \mssec{Energy minima} % drums, metronomes, molecules, tensegrity, whistling e string
        \mpar{Winter}
        For cosmological reasons, the universe is very cold; for statistical
        reasons, we expect everyday objects to imperfectly insulated.  We thus
        encounter many interesting systems near their local energy minima.
        Think of salt crystals or playground swingsets, for instance.  These
        notes study the classical physics of such systems.  We restrict
        ourselves to systems with finitely many degrees of freedom. 

        \mpar{Swingset}
        For example, let's model a swingset as two pendula hanging from a
        shared bar and hence weakly coupled.  The pendula have angles
        $\theta_i$ and angular velocities $\omega_i$ for $i\in \{a,b\}$.  The
        hamiltonian might look something like 
        $$
            \mathcal{H} = 
            \sum_i m \ell^2 \omega_i^2/2 - m g \ell \cos \theta_i
            \,+\, m^2 g^2 (\sin \theta_b - \sin \theta_a)^2/2k
        $$

        \mpar{Crystal}
        As another example, let's model a (one-dimensional) salt crystal as a
        line of $N$ ions of equal mass, spaced $\ell$ apart, and with only
        nearest-neighbor interactions.  The hamiltonian might look something
        like
        $$
            \mathcal{H} = 
                \sum_{0\leq i<N} p_i^2/2m 
                +
                \sum_{0\leq i,i+1<N} k \cdot f(|q_j-q_i|/\ell) 
        $$
        where $f$ is smooth and bounded on the positive reals and minimized
        near $1$.  Perhaps $f$ is Coulomb attraction force plus a repulsive
        term that models Pauli exclusion classically.

        \mpar{Canonical form}
        Now we abstract.  We posit a manifold $M$ equipped with a symplectic
        form $\omega:TM\to T^\star M$ and a smooth hamiltonian $\Hh:M\to\RR$. 
        The vector field $v:M\to TM$ such that $\omega_p(v_p) = (d \Hh)_p$ 
        determines evolution through time.  Near a minimum $p$ of $\Hh$, $\Hh$
        is roughly quadratic.  In fact, we can find a symplectic basis
        for $T_p M$

    \mssec{Spectral asymptotics}
    \mssec{Dispersion}
        \mpar{Waves}
        \mpar{Group velocity}
        \mpar{Waves}
        \mpar{Waves}
    \mssec{Beats}

  \msec{Interactions}
    \mssec{Harmonics}
    \mssec{Catastrophes} % Amplitude dependence
    \mssec{Diagrammatic expansion}
    \mssec{Poincare perturbation}

  \msec{Resonance}
    \mssec{Forcing resonance}
    \mssec{Parametric resonance}
    \mssec{Invariant tori}
    \mssec{Small divisors} % KAM theory

  \msec{Heat}
    \mssec{Thermal conduction} % \___ together dominate internal friction
    \mssec{Thermo-elasticity}  % /
    \mssec{Significance of phase space measure: Boltzmann's law}
    \mssec{Equipartition of energy}

\end{document}


