% author:   sam tenka
% create:   2022-04
% change:   2022-04

%==============================================================================
%=====  0.  DOCUMENT SETTINGS  ================================================
%==============================================================================

%~~~~~~~~~~~~~~~~~~~~~~~~~~~~~~~~~~~~~~~~~~~~~~~~~~~~~~~~~~~~~~~~~~~~~~~~~~~~~~
%~~~~~~~~~~~~~  0.0. About and Beyond this Exposition  ~~~~~~~~~~~~~~~~~~~~~~~~

%---------------------  0.0.0. page geometry  ---------------------------------
\documentclass[11pt]{article}
\usepackage[top=3cm, bottom=3cm, left=3cm, right=3cm]{geometry}

%---------------------  0.0.1. math packages  ---------------------------------
\usepackage{amsmath, amssymb, amsthm, mathtools, bm, moresize}%, euler}

%---------------------  0.0.2. graphics packages  -----------------------------
\usepackage{graphicx, epsdice, xcolor, float, wrapfig, caption}

%~~~~~~~~~~~~~~~~~~~~~~~~~~~~~~~~~~~~~~~~~~~~~~~~~~~~~~~~~~~~~~~~~~~~~~~~~~~~~~
%~~~~~~~~~~~~~  0.1. Header Formatting  ~~~~~~~~~~~~~~~~~~~~~~~~~~~~~~~~~~~~~~~

\definecolor{mblu}{rgb}{0.05, 0.35, 0.70}
\newcommand{\blu}{\color{mblu}}

\definecolor{mbre}{rgb}{0.35, 0.50, 0.65}
\newcommand{\bre}{\color{mbre}}

\definecolor{mgre}{rgb}{0.65, 0.65, 0.60}
\newcommand{\gre}{\color{mgre}}

%---------------------  0.1.0. tidbit headers  --------------------------------
\newcommand{\samtitle} [1]{
  \par\noindent{\Huge \sf \blu #1}
  \vspace{0.3cm}
}

\newcommand{\samquote} [2]{
  \par\noindent{\begin{flushright}
    \scriptsize \gre {\it #1} \\ --- #2
  \end{flushright}}
}

%---------------------  0.1.1. section headers  -------------------------------

\newcommand{\samsection} [1]{
  \vspace{0.5cm}
  \par\noindent{\large \sf \blu #1}
  \vspace{0.3cm}\par
}

\newcommand{\samsubsection}[1]{
  \vspace{0.3cm}
  \par\noindent{\normalsize \sf \bre #1}
  \vspace{0.1cm}\par
}

%---------------------  0.1.2. clear the bibliography's header  ---------------
\usepackage{etoolbox}
\patchcmd{\thebibliography}{\section*{\refname}}{}{}{}

%~~~~~~~~~~~~~~~~~~~~~~~~~~~~~~~~~~~~~~~~~~~~~~~~~~~~~~~~~~~~~~~~~~~~~~~~~~~~~~
%~~~~~~~~~~~~~  0.2. Math Symbols and Blocks  ~~~~~~~~~~~~~~~~~~~~~~~~~~~~~~~~~

%---------------------  0.2.0. probability symbols  ---------------------------

\newcommand{\KL}{\text{KL}}
\newcommand{\EN}{\text{H}}
\newcommand{\note}[1]{{\blu \textsf{#1}}}

%---------------------  0.2.1. double-struck and caligraphic letters  ---------
\newcommand{\Aa}{\mathbb{A}}\newcommand{\aA}{\mathcal{A}}
\newcommand{\Bb}{\mathbb{B}}\newcommand{\bB}{\mathcal{B}}
\newcommand{\Cc}{\mathbb{C}}\newcommand{\cC}{\mathcal{C}}
\newcommand{\Dd}{\mathbb{D}}\newcommand{\dD}{\mathcal{D}}
\newcommand{\Ee}{\mathbb{E}}\newcommand{\eE}{\mathcal{E}}
\newcommand{\Ff}{\mathbb{F}}\newcommand{\fF}{\mathcal{F}}
\newcommand{\Gg}{\mathbb{G}}\newcommand{\gG}{\mathcal{G}}
\newcommand{\Hh}{\mathbb{H}}\newcommand{\hH}{\mathcal{H}}
\newcommand{\Ii}{\mathbb{I}}\newcommand{\iI}{\mathcal{I}}
\newcommand{\Jj}{\mathbb{J}}\newcommand{\jJ}{\mathcal{J}}
\newcommand{\Kk}{\mathbb{K}}\newcommand{\kK}{\mathcal{K}}
\newcommand{\Ll}{\mathbb{L}}\newcommand{\lL}{\mathcal{L}}
\newcommand{\Mm}{\mathbb{M}}\newcommand{\mM}{\mathcal{M}}
\newcommand{\Nn}{\mathbb{N}}\newcommand{\nN}{\mathcal{N}}
\newcommand{\Oo}{\mathbb{O}}\newcommand{\oO}{\mathcal{O}}
\newcommand{\Pp}{\mathbb{P}}\newcommand{\pP}{\mathcal{P}}
\newcommand{\Qq}{\mathbb{Q}}\newcommand{\qQ}{\mathcal{Q}}
\newcommand{\Rr}{\mathbb{R}}\newcommand{\rR}{\mathcal{R}}
\newcommand{\Ss}{\mathbb{S}}\newcommand{\sS}{\mathcal{S}}
\newcommand{\Tt}{\mathbb{T}}\newcommand{\tT}{\mathcal{T}}
\newcommand{\Uu}{\mathbb{U}}\newcommand{\uU}{\mathcal{U}}
\newcommand{\Vv}{\mathbb{V}}\newcommand{\vV}{\mathcal{V}}
\newcommand{\Ww}{\mathbb{W}}\newcommand{\wW}{\mathcal{W}}
\newcommand{\Xx}{\mathbb{X}}\newcommand{\xX}{\mathcal{X}}
\newcommand{\Yy}{\mathbb{Y}}\newcommand{\yY}{\mathcal{Y}}
\newcommand{\Zz}{\mathbb{Z}}\newcommand{\zZ}{\mathcal{Z}}

%---------------------  0.2.2. math environments  -----------------------------
\newtheorem*{qst}{Question}
\newtheorem*{thm}{Theorem}
\newtheorem*{lem}{Lemma}
% ...
\theoremstyle{definition}
\newtheorem*{dfn}{Definition}

%~~~~~~~~~~~~~~~~~~~~~~~~~~~~~~~~~~~~~~~~~~~~~~~~~~~~~~~~~~~~~~~~~~~~~~~~~~~~~~
%~~~~~~~~~~~~~  0.3. Section Headers  ~~~~~~~~~~~~~~~~~~~~~~~~~~~~~~~~~~~~~~~~~


\begin{document}
\samtitle{musical harmony of Bach, Schubert, Hindemith}
  \samquote{... her mind is a very thin soil, laid an inch or two upon very barren rock.}{virginia woolf}

  We study
    the horizontal and vertical pitch relations
  of the art music
    of the German-speaking countries 
    from the 19th and neighboring centuries
  as represented in the work of
    John Bach, George Handel, Louis Beethoven, Frank Schubert, John Brahms, Gus
    Mahler, and Paul Hindemith.
  %
  Our approach is phenomenological: we'll describe without deep explanation
  which arrangements of pitches in time tend in this idiom to sound beautiful
  and expressive.

  \samsection{Pitches and Intervals}

    \samsubsection{just intonation and 53 EDO}
    \samsubsection{}
    \samsubsection{27 basic chords}
      Let us stack three intervals, each a minor third, major third, or fourth,
      on top of a root note --- say, \note{C}.  The 27 combinations are:
      \begin{description}
        \item[\note{C }, \note{Eb}, \note{Gb}, \note{A }]   1 dimin 
        \item[\note{C }, \note{Eb}, \note{Gb}, \note{Bb}]   1 dimin dom7   b3 minor maj6
        \item[\note{C }, \note{Eb}, \note{Gb}, \note{B }]   1 dimin maj7    7 major fla2
            %
        \item[\note{C }, \note{Eb}, \note{G }, \note{Bb}]   1 minor dom7
        \item[\note{C }, \note{Eb}, \note{G }, \note{B }]   1 minor maj7
        \item[\note{C }, \note{Eb}, \note{G }, \note{C }]   1 minor 
            %
        \item[\note{C }, \note{Eb}, \note{Ab}, \note{B }]  b6 minaj
        \item[\note{C }, \note{Eb}, \note{Ab}, \note{C }]  b6 major 
        \item[\note{C }, \note{Eb}, \note{Ab}, \note{Db}]  b6 major sus4
            %%%%%
        \item[\note{C }, \note{E }, \note{G }, \note{Bb}]   1 major dom7
        \item[\note{C }, \note{E }, \note{G }, \note{B }]   1 major maj7
        \item[\note{C }, \note{E }, \note{G }, \note{C }]   1 major 
            %
        \item[\note{C }, \note{E }, \note{Ab}, \note{B }]   1 augme maj7    3 major min6
        \item[\note{C }, \note{E }, \note{Ab}, \note{C }]   1 augme 
        \item[\note{C }, \note{E }, \note{Ab}, \note{Db}]   1 augme fla2   b2 minor maj7 
            %
        \item[\note{C }, \note{E }, \note{A }, \note{C }]   6 minor
        \item[\note{C }, \note{E }, \note{A }, \note{Db}]   6 minaj 
        \item[\note{C }, \note{E }, \note{A }, \note{D }]   6 minor sus4
            %%%%%
        \item[\note{C }, \note{F }, \note{Ab}, \note{B }]   4 minor sha4     
        \item[\note{C }, \note{F }, \note{Ab}, \note{C }]   4 minor
        \item[\note{C }, \note{F }, \note{Ab}, \note{Db}]   4 minor min6   b2 major maj7 (neapolitan/tonic) 
            %
        \item[\note{C }, \note{F }, \note{A }, \note{C }]   4 major 
        \item[\note{C }, \note{F }, \note{A }, \note{Db}]   4 major min 
        \item[\note{C }, \note{F }, \note{A }, \note{D }]   4 major maj6    2 minor dom7
            %
        \item[\note{C }, \note{F }, \note{Bb}, \note{Db}]   4 susp4 min6   b7 minaj
        \item[\note{C }, \note{F }, \note{Bb}, \note{D }]   4 susp4 maj6    1 susp4 dom7
        \item[\note{C }, \note{F }, \note{Bb}, \note{Eb}]   4 susp4 dom7    
      \end{description}


\end{document}



